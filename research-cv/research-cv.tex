%Research CV template for academics and researchers
%@TheDoctorRAB
\documentclass[letter,oneside,english,10pt,headinclude,headsepline]{amsart}
\usepackage[margin=1in]{geometry}
\usepackage{times}
\usepackage{enumerate}
\usepackage{enumitem}
\usepackage{bold-extra}
\usepackage{bibentry}
\usepackage{filecontents}
\usepackage[yyyymmdd]{datetime} %set preferred date format
\renewcommand{\dateseparator}{.} %set preferred date separator

\makeatletter %makes section in block caps with large font \section*{}
\renewcommand
\section{\@startsection{section}{1}%
  \z@{1.5\linespacing}{.25\linespacing}%
  {\bfseries\scshape\large\raggedright}}

\makeatletter %makes section in block caps with normal font \subsection*{}
\renewcommand
\subsection{\@startsection{subsection}{1}%
  \z@{.75\linespacing}{.25\linespacing}%
  {\bfseries\scshape\raggedright}}

\newcommand*\sq{\mathbin{\vcenter{\hbox{\rule{.3ex}{.3ex}}}}} %makes a small square as a separator $\sq$

\usepackage{fancyhdr} 
\pagestyle{fancy}
\fancyhf{} %page number on bottom right
\lhead{\scriptsize Title Name} %title and name
\chead{\scriptsize \textit{Vita}} %maybe something more specific here if desired
\rhead{\scriptsize \today} %date
\rfoot{\thepage} %page number

\title{Name}

%peer reviewed journal papers
\begin{filecontents}{peer-reviewed.bib} 
    @article{
        pr1,
        author = {},
        journal = {},
        pages = {},
        title = {{}},
        volume = {},
        year = {}
    }
\end{filecontents}

%books and book contributions
\begin{filecontents}{book-contribution.bib}
    @book{
        bk1,
        author = {},
        title = {{}},
        year = {},
        publisher = {{}}
        isbn = {}
    }
\end{filecontents}

%conference proceedings
\begin{filecontents}{conference-proceedings.bib}
    @conference{
        cf1,
        author = {},
        title = {{}},
        year = {},
        organization = {},
        address = {}
    }
\end{filecontents}

%reports, misc, non-refeered
\begin{filecontents}{reports.bib}
    @misc{
        rp1,
        author = {},
        title = {{}},
        institution = {},
        year = {}
    }
\end{filecontents}

%conference presentations - not the same as invited talks 
\begin{filecontents}{conference-presentations.bib}
    @conference{
        cf1,
        author = {},
        title = {{}},
        year = {},
        organization = {},
        address = {}
    }
\end{filecontents}

\begin{document}

%.bib files to create
\nobibliography{peer-reviewed,book-contribution,conference-proceedings,reports,conference-presentations} 

%.bst file
\bibliographystyle{ieeetr} %custom bst file

\thispagestyle{fancy}

\begin{center}{
        \textsc{ %block caps font
            \textbf{
                {\large Name}\\
                Title \\
                Institution $\sq$ Department
                Address
                email $\sq$ twitter $\sq$ phone 
            }
        }
    }
\end{center}

\noindent\makebox[\linewidth]{\rule{\textwidth}{0.5pt}} %horizontal line spanning margins

\section*{Education}
\noindent\textbf{University}\\
\textbf{Doctor of Philosophy $\sq$ Major} \hfill{\textit{Date}}
\begin{itemize}[leftmargin=*,topsep=0pt,itemsep=-1ex,partopsep=1ex,parsep=1ex]
    \item[]\textit{Dissertation: }
    \item[]Examination fields: 
    \item[]Committee: 
\end{itemize}
\vspace*{.75\baselineskip}

\noindent\textbf{University}\\
\textbf{Master of Science $\sq$ Major} \hfill{\textit{Date}}
\begin{itemize}[leftmargin=*,topsep=0pt,itemsep=-1ex,partopsep=1ex,parsep=1ex]
    \item[]\textit{Thesis: }
    \item[]Advisor: %committee
\end{itemize}
\vspace*{.75\baselineskip}

\noindent\textbf{University}\\
\textbf{Bachelor of Science $\sq$ Major and distinction(s)} \hfill{\textit{}}
\begin{itemize}[leftmargin=*,topsep=0pt,itemsep=-1ex,partopsep=1ex,parsep=1ex]
    \item[]\textit{Capstone: }
    \item[] Advisor: Leo M. Bobek
\end{itemize}

%academic positions 
\section*{Research Experience}
\noindent\textbf{University of Idaho $\sq$ Idaho Falls Center for Higher Education $\sq$ Nuclear Engineering Program}\\
\textbf{Assistant Professor} \hfill{\textit{July 2015 - }}\\
Advanced fuel cycle analysis; risk assessment; neutronics; nonproliferation; scientific computing
\begin{itemize}[leftmargin=*,topsep=0pt,itemsep=-1ex,partopsep=1ex,parsep=1ex]
    \item[]\indent\indent Coordinator - NPP Decommissioning and Used Fuel Management Certificate \hfill{\textit{August 2019 - }}
    \item[]\indent\indent Affiliate - Boise State University Energy Policy Institute \hfill{\textit{August 2019 - }}
    \item[]\indent\indent State of Idaho Professional Engineer, Faculty Restricted \hfill{\textit{October 2019 - }}
\end{itemize}
\vspace*{.75\baselineskip}

\noindent\textbf{University of California-Berkeley $\sq$ Department of Nuclear Engineering}\\
\textbf{Postdoctorate Researcher} \hfill{\textit{February 2009 - July 2012}}\\
A safeguards methodology was developed for remotely-handled nuclear materials facilities that proposes functional design components and a risk-informed framework in an effort to integrate safeguards with safety and security into facility design.
\vspace*{.75\baselineskip}

\noindent\textbf{The University of Tokyo $\sq$ Department of Nuclear Engineering/Management}\\
\textbf{Research Associate} \hfill{\textit{August 2007 - January 2009}}\\
Mathematical models for mass transport in the engineered barrier system of a high-level nuclear waste repository are established for bentonite extrusion and radionuclide transport under different environmental conditions to assess potential impacts on performance assessment.
\vspace*{.75\baselineskip}

\noindent\textbf{University of California-Berkeley $\sq$ Department of Nuclear Engineering}\\
\textbf{Postdoctorate Researcher} \hfill{\textit{February 2007 - April 2007}}\\
Mathematical modeling of mass transport in the engineered barrier system of a high-level nuclear waste repository continued, to assess the potential of extruding bentonite to confine radionuclides.
\vspace*{.75\baselineskip}

\noindent\textbf{University of California-Berkeley $\sq$ Department of Nuclear Engineering}\\
\textbf{Doctoral Candidate} \hfill{\textit{August 2005 - December 2006}}\\
The dissertation thesis focused on the mathematical modeling of mass transport in the engineered barrier system of a high-level nuclear waste repository. A mass transport model was established for radionuclides in a porous medium, bentonite extrusion model in a planar fracture to assess the potential to confine radionuclides.

\section*{Related Research Experience}
\noindent\textbf{Earth Sciences Division $\sq$ Lawrence Berkeley National Laboratory}\\
\textbf{Intern} \hfill{\textit{July 2001 - February 2002}}\\
Conducted data analysis for chaotic modeling of water flow in the unsaturated zone.
\vspace*{.75\baselineskip}

\noindent\textbf{Leslie C. Wilbur Nuclear Reactor Facility $\sq$ Worcester Polytechnic Institute}\\
\textbf{Major Qualifying Project} \hfill{\textit{August 1995 - May 1996}}\\
The Major Qualifying Project synthesizes previous undergraduate studies to solve problems or perform tasks in the major field and communicate results effectively. For this project, a logic algorithm was developed for reactivity derived from point kinetics equations and a programmable logic controller was modified for real time collection of data during nuclear reactor operation for use with experimental coursework and maintenance procedures.

\section*{Related Professional Experience}
\noindent\textbf{Leslie C. Wilbur Nuclear Reactor Facility $\sq$ Worcester Polytechnic Institute}\\
\textbf{Assistant Radiation Safety Officer} \hfill{\textit{October 1996 - August 1999}}
\begin{itemize}[leftmargin=*,topsep=0pt,itemsep=-1ex,partopsep=1ex,parsep=1ex]
    \item[]Responded to emergencies concerning incidents involving radioactive material
    \item[]Maintained records of radiation exposures and dosimetry to campus personnel
    \item[]Maintained the full and current inventory of radioisotopes
    \item[]Performed calibrations for radiation detectors
    \item[]Instructed all radiological and nuclear reactor laboratory safety training
    \item[]Performed secretarial duties for Radiation, Health, and Safety Committee
\end{itemize}
\vspace*{.75\baselineskip}

\noindent\textbf{Leslie C. Wilbur Nuclear Reactor Facility $\sq$ Worcester Polytechnic Institute}\\
\textbf{Senior Nuclear Reactor Operator \#70145} \hfill{\textit{August 1994 - August 1999}}
\begin{itemize}[leftmargin=*,topsep=0pt,itemsep=-1ex,partopsep=1ex,parsep=1ex]
    \item[]Operated the university nuclear reactor for experimental research projects
    \item[]Trained reactor operator license candidates
    \item[]Performed corrective and preventative maintenance of reactor systems
    \item[]Modified operator training program to a computer based system
\end{itemize}

\section*{Awarded Projects} %chronologically
\begin{enumerate}[leftmargin=*,topsep=0pt,itemsep=.15ex,partopsep=1ex,parsep=1ex]
    \item Lee Ostrom (PI), Richard N. Christensen, R. A. Borrelli, Haiyan Zhao (co-PIs) - University of Idaho. ORED Fall 2019 EIS: Portable XFR for use in supporting material research. ORED Equipment and Infrastructure Support. \textbf{\$40,000.} 2019.12.01 - 2020.11.30
    \item R. A. Borrelli (PI) - University of Idaho, Mark. D. DeHart (co-PI) - Idaho National Laboratory. Application and enhancement of MAMMOTH depletion capabilities. \textbf{\$36,894.} 2020.01.13 - 2020.12.31
    \item Richard N. Christensen (PI), R. A. Borrelli, Michael G. McKellar, Michael Haney, David Arcilesi (co-PIs) - University of Idaho, Richard Jacobson (co-PI) Idaho State University. NuScale Simulator at the Center for Advanced Energy Studies. Department of Energy Scientific Infrastructure Support for Consolidated Innovative Nuclear Research. \textbf{\$285,763.} 2019.10.01 - 2020.09.30
    \item R. A. Borrelli (PI) - University of Idaho, Dennis D. Keiser, Jr., (co-PI) - Idaho National Laboratory. Graduate Research Assistantship: Connecting U-Mo Fuel Processing, Microstructure, and Irradiation Performance. \textbf{\$52,070.57.} 2018.10.01-2019.09.30
    \item R. A. Borrelli (PI), Richard N. Christensen (co-PI) - University of Idaho, Brian T. Jaques (co-PI) - Boise State University, Piyush Sabharwall (co-PI) - Idaho National Laboratory, Mark Delligatti (co-PI) - Table Rock, LLC, Sakae Casting USA, LLC (co-PI). Modeling and design of borated aluminium cask for used fuel cooling. Idaho Global Entrepreneurial Mission (IGEM) - Idaho Commerce, \textbf{\$237,898.} 2018.01.01-2019.05.31
    \item R. A. Borrelli (PI) - University of Idaho, Dennis D. Keiser, Jr., (co-PI) - Idaho National Laboratory. Graduate Research Assistantship: Connecting U-Mo Fuel Processing, Microstructure, and Irradiation Performance. \textbf{\$36,180.} 2017.11.01-2018.05.31
    \item R. A. Borrelli (PI), Lee Ostrom (Senior Advisor) - University of Idaho, Stephen G. Johnson (Senior Advisor) - Idaho National Laboratory. Performance assessment of americium as fuel in radioisotope thermoelectric generators for deep space exploration. Idaho NASA EPSCoR Research Initiation Grant. \textbf{\$55,000.} 2017.08.01-2018.04.30
    \item Kelley Verner (PI), R. A. Borrelli, Marc T. Skinner, Emma Redfoot, Jieun Lee, Seth Dustin, John Peterson (co-PIs) - University of Idaho. Increasing the Go-on Rate in Southeast Idaho Through the Nexus of Food, Energy, and Water. University of Idaho Vandals Big Ideas Project. \textbf{\$23,984.} 2017.07.01-2018.06.30
    \item R. A. Borrelli (PI) - University of Idaho, Jason Hales (co-PI) - Idaho National Laboratory. Graduate Research Assistantship: Idaho National Laboratory Code Documentation. \textbf{\$35,435.} 2016.10.01-2017.06.30
    \item Vivek Utgikar (PI), Fatih Aydogan, Krishnan Raja, Raghunath Kanakala, R. A. Borrelli, Haiyan Zhao, Matthew Swenson (co-PIs) - University of Idaho. University of Idaho Nuclear Engineering Faculty Development Program. United States Regulatory Commission Faculty Development Grant. \textbf{\$434,048.} 2015.09.29 - 2019.09.30
\end{enumerate}

\section*{Equipment Acquisitions}
\noindent University of Idaho, Idaho State University, Idaho National Laboratory. Western Services Corporation Pressurized Water Nuclear Reactor Simulator.

\section*{Thesis Committees}
\subsection*{Major Professor}
\noindent\textit{Current}
\begin{itemize}[leftmargin=*,topsep=0pt,itemsep=-1ex,partopsep=1ex,parsep=1ex]
    \item[]Kelley Verner, Ph.D. Candidate - Materials science and nuclear fuels analysis
    \item[]Joseph Christensen, Ph.D. - Criticality safety modeling
    \item[]John Peterson, Ph.D - Nuclear cybersecurity
    \item[]J. Seth Dustin, MS - Satellite RTG performance assessment
    \item[]Olin Calvin, Ph.D. - Enhancement of MAMMOTH depletion capabilities
    \item[]Teyen Widdicombe, Ph.D. - RTG satellite design
\end{itemize}
\vspace*{.75\baselineskip}

\noindent\textit{Graduates}
\begin{itemize}[leftmargin=*,topsep=0pt,itemsep=-1ex,partopsep=1ex,parsep=1ex]
    \item[]John Peterson, MS - Molten salt reactor neutronics design
    \item[]Emma Redfoot, MS - Allocating heat and electricity in a nuclear renewable hybrid energy system coupled with a water purification system
    \item[]Jieun Lee, MS - Risk-informed safeguards of pyroprocessing for advanced nuclear fuel concepts
    \item[]Trevor MacLean, MEng - Cybersecurity modeling of non-critical nuclear power plant instrumentation
    \item[]Malachi Tolman, MEng - INL BISON code documentation
\end{itemize}

\subsection*{Member}
\noindent\textit{Graduates}
\begin{itemize}[leftmargin=*,topsep=0pt,itemsep=-1ex,partopsep=1ex,parsep=1ex]
    \item[]D. Devin Imholte, MS - Conceptual design of the Advanced Test Reactor non-destructive examination system
    \item[]John Biersdorf, MEng - Precipitation changes to Idaho National Laboratory over time
    \item[]Winfred Sowah, MS - Thermal behavior of cold plated storage cask for light water reactor nuclear fuels
    \vspace*{.05in}
    \item[]\textit{Idaho State University}
    \item[]Pedro Mena, MS - Reactor transient classification using machine learning 
\end{itemize}

\section*{Refereed Journal Publications}
\begin{enumerate}[leftmargin=*,topsep=0pt,itemsep=.15ex,partopsep=1ex,parsep=1ex]
    \item\bibentry{pr1}.
    \item\bibentry{bor20b}.
    \item\bibentry{bor20c}.
    \item\bibentry{bor19a}.
    \item\bibentry{bor19b}.
    \item\bibentry{bor19c}.
    \item\bibentry{bor18a}.
    \item\bibentry{bor17a}.
    \item\bibentry{bor17b}.
    \item\bibentry{bor16a}.
    \item\bibentry{bor14a}.
    \item\bibentry{bor14b}.
    \item\bibentry{bor14c}.
    \item\bibentry{bor13a}.
    \item\bibentry{bor13b}.
    \item\bibentry{bor11a}.
    \item\bibentry{bor08a}.
    \item\bibentry{bor08b}.
\end{enumerate}

\section*{Book Contribution}
\begin{enumerate}[leftmargin=*,topsep=0pt,itemsep=.15ex,partopsep=1ex,parsep=1ex]
    \item\bibentry{bkbor19}.
    \item\bibentry{bkbor15}.
\end{enumerate}

\section*{Refereed Conference Proceedings}
\begin{enumerate}[leftmargin=*,topsep=0pt,itemsep=.15ex,partopsep=1ex,parsep=1ex]
    \item\bibentry{cfbor19a}.
    \item\bibentry{cfbor19b}.
    \item\bibentry{cfbor19c}.
    \item\bibentry{cfbor19d}.
    \item\bibentry{cfbor19e}.
    \item\bibentry{cfbor18a}.
    \item\bibentry{cfbor18b}.
    \item\bibentry{cfbor18c}.
    \item\bibentry{cfbor18d}.
    \item\bibentry{cfbor17a}.
    \item\bibentry{cfbor17b}.
    \item\bibentry{cfbor17c}.
    \item\bibentry{cfbor17d}.
    \item\bibentry{cfbor17e}.
    \item\bibentry{cfbor16a}.
    \item\bibentry{cfbor13a}.
    \item\bibentry{cfbor13b}.
    \item\bibentry{cfbor12a}.
    \item\bibentry{cfbor11a}.
    \item\bibentry{cfbor10a}.
    \item\bibentry{cfbor10b}.
    \item\bibentry{cfbor09a}.
    \item\bibentry{cfbor09b}.
    \item\bibentry{cfbor08a}.
    \item\bibentry{cfbor07a}.
    \item\bibentry{cfbor07b}.
    \item\bibentry{cfbor06a}.
    \item\bibentry{cfbor96a}.
\end{enumerate}

\section*{Non-Refereed Technical Reports}
\begin{enumerate}[leftmargin=*,topsep=0pt,itemsep=.15ex,partopsep=1ex,parsep=1ex]
    \item\bibentry{rpbor20a}.
    \item\bibentry{rpbor19a}.
    \item\bibentry{rpbor18a}.
    \item\bibentry{rpbor15a}.
    \item\bibentry{rpbor12a}.
    \item\bibentry{rpbor12b}.
    \item\bibentry{rpbor11a}.
    \item\bibentry{rpbor10a}.
    \item\bibentry{rpbor10b}.
    \item\bibentry{rpbor09a}.
    \item\bibentry{rpbor07a}.
    \item\bibentry{rpbor07b}.
    \item\bibentry{rpbor07c}.
    \item\bibentry{rpbor06a}.
\end{enumerate}

\section*{Conference Presentations}
\begin{enumerate}[leftmargin=*,topsep=0pt,itemsep=.15ex,partopsep=1ex,parsep=1ex]
    \item\bibentry{cfpbor19a}.
    \item\bibentry{cfpbor19b}.
    \item\bibentry{cfpbor19c}.
    \item\bibentry{cfpbor19d}.
    \item\bibentry{cfpbor19e}.
    \item\bibentry{cfpbor19f}.
    \item\bibentry{cfpbor18a}.
    \item\bibentry{cfpbor18b}.
    \item\bibentry{cfpbor18c}.
    \item\bibentry{cfpbor18d}.
    \item\bibentry{cfpbor18e}.
    \item\bibentry{cfpbor17a}.
    \item\bibentry{cfpbor17b}.
    \item\bibentry{cfpbor17c}.
    \item\bibentry{cfpbor17d}.
    \item\bibentry{cfpbor17e}.
    \item\bibentry{cfpbor16a}.
    \item\bibentry{cfpbor13a}.
    \item\bibentry{cfpbor13b}.
    \item\bibentry{cfpbor12a}.
    \item\bibentry{cfpbor11a}.
    \item\bibentry{cfpbor10a}.
    \item\bibentry{cfpbor10b}.
    \item\bibentry{cfpbor09a}.
    \item\bibentry{cfpbor09b}.
    \item\bibentry{cfpbor08a}.
    \item\bibentry{cfpbor07a}.
    \item\bibentry{cfpbor07b}.
    \item\bibentry{cfpbor06a}.
\end{enumerate}

\section*{Teaching Experience}
\noindent\textbf{University of Idaho $\sq$ Idaho Falls Center for Higher Education $\sq$ Nuclear Engineering Program}\\
\textbf{Assistant Professor}\\ 
\textit{NE502: Historical examinations of heterogeneity in nuclear criticality safety \hfill{Spring 2020}}\\
\hspace*{\fill}\textit{directed study}\\
Using published nuclear criticality safety references, this course will conduct an examination of the effect of heterogeneity with respect to nuclear criticality safety. As part of the history of nuclear criticality safety, a number of critical experiments have been conducted using arrays of fissile material in an attempt to quantify the effect of heterogeneity in the determination of safe handling limits. In these attempts, a number of low-enriched experiments were examined and converted in their critical dimensions to establish a useful baseline from which other limits could be extrapolated. It is of interest to the field of nuclear criticality safety to improve the understanding of the effect of heterogeneity on the critical characteristics of multiplying fissile systems, particularly in the range of intermediate enrichment. It is of further interest to the field that a methodology be established which can be used to evaluate these types of systems for practical nuclear criticality safety applications, especially the development of nuclear criticality safety limits. To that end, this course will identify relevant data in the literature used to generate historical experiments and apply MCNP to model these experiments in order to understand and evaluate assumptions and restrictions raised in the experiments.
\vspace*{.75\baselineskip}

\noindent\textbf{University of Idaho $\sq$ Idaho Falls Center for Higher Education $\sq$ Nuclear Engineering Program}\\
\textbf{Assistant Professor}\\ 
\textit{NE502: Heterogeneity comparisons of intermediate enrichment uranium in critical systems \hfill{Spring 2020}}\\
\hspace*{\fill}\textit{directed study}\\
Using the International Criticality Safety Benchmark Evaluation Project (ICSBEP) handbook of evaluated critical experiments, conduct an examination of the effect of heterogeneity in intermediate-enrichment uranium systems. The ICSBEP handbook contains descriptions and evaluations of critical experiments conducted in facilities throughout the world. The reach of the handbook stretches to the beginning of the history of nuclear engineering. In those evaluations, a variety of methods have been discussed and described which convert a system of known or unknown degree of heterogeneity into a simplified homogeneous model, which is a traditional method for examining the critical characteristics of multiplying systems. The specific methodology for this conversion varies between evaluations and is explained in some cases, where other cases are less descriptive. It is of interest to the field of nuclear criticality safety to improve the understanding of the effect of heterogeneity on the critical characteristics of multiplying fissile systems, particularly in the range of intermediate enrichment. It is of further interest to the field that a methodology be established which can be used to evaluate these types of systems for practical nuclear criticality safety applications, especially the development of nuclear criticality safety limits. This course will identify and evaluate relevant benchmark experiments for heterogeneity effects using MCNP to establish new baseline models from the benchmark handbook.
\vspace*{.75\baselineskip}

\noindent\textbf{University of Idaho $\sq$ Idaho Falls Center for Higher Education $\sq$ Nuclear Engineering Program}\\
\textbf{Assistant Professor}\\ 
\textit{NE502: Nuclear integrated energy systems \hfill{Spring 2020}}\\
\hspace*{\fill}\textit{directed study}\\
Nuclear renewable hybrid energy systems enable a nuclear reactor to load follow with a renewable energy source. These must be designed to distribute energy dynamically by supplying electricity to the grid while using either thermal or electrical energy for industrial applications. This takes advantage of the flexible distribution of electricity or heat to maximize profit. The industrial process serves as a load sink for the excess heat or electricity produced by the nuclear reactor. Economic challenges to the current United States nuclear light water reactor (LWR) fleet have led to early plant closures. While LWRs primarily deliver baseload electricity, there is no reason why nuclear energy produced by these reactors cannot be used to provide energy to a range of industrial applications. This directed study course will identify feasible systems, products, and commodities that could be produced by existing nuclear plants. As part of this, cost and potential profitability will be analyzed within the context of market structures and grid reliability.
\vspace*{.75\baselineskip}

\noindent\textbf{University of Idaho $\sq$ Idaho Falls Center for Higher Education $\sq$ Nuclear Engineering Program}\\
\textbf{Assistant Professor}\\ 
\textit{NE527: Nuclear material storage, transport, disposal \hfill{Spring 2020}}\\
There is a wide range of nuclear materials that are stored, transported and disposed of each day. The materials include medical radioisotopes, new fuel pellets, used fuel, and industrial radioisotopes. This course will cover the regulations that govern nuclear material storage, transportation and disposal, as well as the engineering requirements and practical aspects of handling these materials.
\vspace*{.75\baselineskip}

\noindent\textbf{University of Idaho $\sq$ Idaho Falls Center for Higher Education $\sq$ Nuclear Engineering Program}\\
\textbf{Assistant Professor}\\ 
\textit{NE535: Nuclear Criticality Safety I \hfill{Spring 2020}}\\
This course applies uses the foundation of applied nuclear physics to develop and explain the international and domestic rules and practices that are used to prevent inadvertent criticality in fuel cycle applications such as used fuel storage and processing.
\vspace*{.75\baselineskip}

\noindent\textbf{University of Idaho $\sq$ Idaho Falls Center for Higher Education $\sq$ Nuclear Engineering Program}\\
\textbf{Assistant Professor}\\ 
\textit{NE585: Nuclear Fuel Cycle Analysis \hfill{Fall 2017}}\\
This course presents the nuclear fuel cycle can as an holistic system with components related in many complex ways. This course focuses on systems analysis of components that comprise the nuclear fuel cycle to understand the contemporary challenges facing nuclear energy.
\vspace*{.75\baselineskip}

\noindent\textbf{University of Idaho $\sq$ Idaho Falls Center for Higher Education $\sq$ Nuclear Engineering Program}\\
\textbf{Assistant Professor}\\ 
\textit{TM529: Risk Assessment \hfill{Spring 2019; 2018; 2017; 2016}}\\
This course is designed to provide students with an understanding of how to perform a comprehensive risk assessment applicable to a wide variety of engineering problems in many different disciplines. The course will focus on failure mode and effect analysis, fault tree analysis, probabilistic risk analysis, and human reliability analysis. The course will also cover fundamental probability and statistics content.
\vspace*{.75\baselineskip}

\noindent\textbf{University of Idaho $\sq$ Idaho Falls Center for Higher Education $\sq$ Nuclear Engineering Program}\\
\textbf{Assistant Professor}\\ 
\textit{NE450: Principles of Nuclear Engineering \hfill{Fall 2019; 2018; 2017; 2016; 2015}}\\
In this course, an overview of fundamental nuclear engineering principles and how these are practically applied to contemporary, nuclear engineering problems will be presented.  The topics covered in this course include: nuclear physics and reactions, materials science, radiation protection, energy production, fuel cycle analysis, advanced reactor design, fusion, nonproliferation, back-end management, and risk assessment and safety.  Throughout the course, the ethical considerations with regards to engineering problems within these fields will also be discussed.
\vspace*{.75\baselineskip}

\noindent\textbf{Diablo Valley College $\sq$ Department of Architecture and Engineering}\\
\textbf{Adjunct Professor}\\ 
\textit{ENGIN110: Introduction to Engineering \hfill{Spring 2015; 2014; 2013; Fall 2014}}\\
This course introduces students to fundamental engineering principles. Students learn how these are applied to contemporary engineering problems through laboratory exercises, homework assignments, design projects, interviews with professional engineers, and field trips to engineering companies. Topics include: materials science, risk assessment and safety, critical problem-solving, engineering analysis, engineering design processes, project development, engineering software, and presentation tools. The role of the engineer in society, professionalism, and engineering ethics are major themes. The emphasis is on creative problem-solving, teamwork, and effective communication, both in presentation and writing.
\vspace*{.75\baselineskip}

\noindent\textbf{University of California-Berkeley $\sq$ Department of Nuclear Engineering}\\
\textbf{Instructor}\\
\textit{NE92: Issues in Nuclear Science and Engineering \hfill{Fall 2011}}\\
This course provides undergraduate students with an overview of the nuclear engineering profession, including fundamental nuclear engineering principles and how these are practically applied to nuclear engineering problems. The topics covered include nuclear physics and reactions, materials science, radiation protection, energy production, fuel cycle analysis, advanced reactor design, fusion, nonproliferation, back-end management, risk assessment, and safety. Throughout the course, the ethical considerations concerning engineering problems within these fields are also addressed through a comprehensive, student-driven course project.
\vspace*{.75\baselineskip}

\noindent\textbf{University of California-Berkeley $\sq$ Department of Nuclear Engineering}\\
\textbf{Instructor}\\
\textit{NE375: Teaching Techniques in Nuclear Engineering \hfill{Fall 2011; 2010}}\\
This course acquaints graduate student instructors (GSIs) with teaching techniques for courses in the Department of Nuclear Engineering. The GSI will have several duties far beyond grading assignments and/or examinations: conducting discussion sessions, review lectures, or laboratory experiments. The GSI, therefore, needs to develop the appropriate tools to use when facing these pedagogical challenges. Three students from the 2010 course received the Outstanding Graduate Student Instructor Award given by the UC-Berkeley Graduate Student Instructor Teaching \& Resource Center.
\vspace*{.75\baselineskip}

\noindent\textbf{The University of Tokyo $\sq$ Department of Nuclear Engineering/Management}\\
\textbf{Part time lecturer}\\
\textit{Technical English for Scientists \hfill{Winter, Summer 2008; Winter 2007}}\\
This course provided the opportunity for non-native English speaking students to develop technical communication skills; i.e., presenting scientific and technical material to an informed audience at an international conference. In this course, the ‘assertion evidence design’ concept for technical presentation of scientific topics was applied to student research interests. Transmutable skills focused on the professional communication of scientific research in various public speaking formats and a comfortable familiarity with the English language to establish a stronger foundation for technical writing.

\section*{Related Teaching Experience}
\noindent\textbf{University of California-Berkeley $\sq$ Department of Nuclear Engineering}\\
\textbf{Graduate Student Instructor}\\
\textit{NE375: Teaching Techniques in Nuclear Engineering \hfill{Fall 2006}}
\begin{itemize}[leftmargin=*,topsep=0pt,itemsep=-1ex,partopsep=1ex,parsep=1ex]
    \item[]Coordinated guest lecturers from departments within the College of Engineering
    \item[]Prepared lectures for effective teaching strategies for the undergraduate classroom
    \item[]Assessed individual student technical presentations and overall course grading
\end{itemize}
\vspace*{.75\baselineskip}

\noindent\textit{E124: Ethics and the Impact of Technology on Society \hfill{Spring 2006; 2005; 2004}}
\begin{itemize}[leftmargin=*,topsep=0pt,itemsep=-1ex,partopsep=1ex,parsep=1ex]
    \item[]Conducted multiple discussion sections on a weekly basis and review lectures
    \item[]Supervised research projects based on current, ethical and scientific issues 
    \item[]Assessed individual student presentations, projects, and overall course grading
\end{itemize}
\vspace*{.75\baselineskip}

\noindent\textit{IDS110: Introduction to Computing \hfill{Fall 2004}}
\begin{itemize}[leftmargin=*,topsep=0pt,itemsep=-1ex,partopsep=1ex,parsep=1ex]
    \item[]Conducted multiple laboratory sessions on a weekly basis
    \item[]Supervised undergraduate research projects focused on web based education
    \item[]Assessed laboratory assignments and project grading
\end{itemize}
\vspace*{.75\baselineskip}

\noindent\textit{NE92: Issues in Nuclear Science and Engineering \hfill{Spring 2002; 2000}}
\begin{itemize}[leftmargin=*,topsep=0pt,itemsep=-1ex,partopsep=1ex,parsep=1ex]
    \item[]Coordinated guest lecturers from Department of Nuclear Engineering
    \item[]Conducted review lectures
    \item[]Developed examinations and homework assignments
    \item[]Assessed overall course grading
\end{itemize}
\vspace*{.75\baselineskip}

\noindent\textit{NE275: Principles and Methods of Risk Analysis \hfill{Fall 2001}}
\begin{itemize}[leftmargin=*,topsep=0pt,itemsep=-1ex,partopsep=1ex,parsep=1ex]
    \item[]This graduate course requires a deeper understanding of the subject matter, due to the student body. The course was one of three in the curriculum with the highest credit load. The main responsibility in this was to advise and grade semester projects and presentations based on risk assessments of engineering systems.
\end{itemize}
\vspace*{.75\baselineskip}

\noindent\textbf{University of California-Berkeley $\sq$ Department of Nuclear Engineering}\\
\textbf{Reader}\\
\noindent\textit{NE150: Introduction to Nuclear Reactor Theory \hfill{Spring 2003}}\\
\noindent\textit{NE104: Radiation Detection and Nuclear Instrumentation Laboratory \hfill{Fall 2002}}\\
\noindent\textit{NE107: Introduction to Imaging \hfill{Spring 2001}}\\
\noindent\textit{NE120: Nuclear Materials \hfill{Fall 2000}}
\begin{itemize}[leftmargin=*,topsep=0pt,itemsep=-1ex,partopsep=1ex,parsep=1ex]
    \item[]Supervised laboratory sessions
    \item[]Assessed examinations, homework assignments, laboratory reports, final grades
    \item[]Conducted review lectures
\end{itemize}

\section*{Professional Service}
\subsection*{University of Idaho}
\begin{itemize}[leftmargin=*,topsep=0pt,itemsep=-1ex,partopsep=1ex,parsep=1ex]
    \item[]Faculty Advisor - American Nuclear Society, University of Idaho Student Section
    \item[]Member - Graduate Faculty, University of Idaho
    \item[]Member - NRC Student Fellowship Oversight Committee, Nuclear Engineering Program
    \item[]Member - Nuclear Engineering Program Admissions Committee
    \item[]Member - Idaho Falls Center for Higher Education Safety Committee
\end{itemize}

\subsection*{Academic Search Committees}
\begin{itemize}[leftmargin=*,topsep=0pt,itemsep=-1ex,partopsep=1ex,parsep=1ex]
    \item[]\textit{2018}
    \item[]Nuclear Engineering Faculty, Idaho State University, Idaho Falls Polytechnic Institute
            \vspace{.05in}
    \item[]\textit{2017}
    \item[]Mechanical/Nuclear Engineering Faculty, University of Idaho
\end{itemize}

\subsection*{Peer Reviewer}
\begin{itemize}[leftmargin=*,topsep=0pt,itemsep=-1ex,partopsep=1ex,parsep=1ex]
    \item[]\textit{Regular periods}
    \item[]American Nuclear Society Annual and Winter Meetings 
    \item[]IEEE Transactions on Nuclear Science 
    \item[]American Nuclear Society Fuel Cycle and Waste Management Division John Randall Scholarship 
    \item[]Nuclear Science User Facilities 
    \item[]Advances in Engineering Software 
    \item[]Nuclear Engineering and Technology 
    \item[]Annals of Nuclear Energy 
    \item[]Progress in Nuclear Energy 
    \item[]Energy Science \& Engineering 
    \item[]Nuclear Engineering and Design 
    \item[]International Journal of Nuclear Energy 
            \vspace{.05in}
    \item[]\textit{2020}
    \item[]American Nuclear Society Student Conference - North Carolina State University
            \vspace{.05in}
    \item[]\textit{2019}
    \item[]John Wiley \& Sons, Inc. 
    \item[]Khalifa University of Science and Technology 
    \item[]American Nuclear Society Idaho Section - Idaho High School Essay Contest 
    \item[]American Nuclear Society Student Conference - Virginia Commonwealth University  
        \vspace{.05in}
    \item[]\textit{2018}
    \item[]American Nuclear Society Idaho Section - Idaho High School Essay Contest 
    \item[]USDOE - SBIR/STTR Phase I Release 2 
    \item[]American Nuclear Society Student Conference - University of Florida  
        \vspace{.05in}
    \item[]\textit{2017}
    \item[]American Nuclear Society Idaho Section - Idaho High School Essay Contest 
    \item[]USDOE - SBIR/STTR Phase II Release 2 
    \item[]USDOE - SBIR/STTR Phase II Release 1 
    \item[]USDOE - Office of Nuclear Energy, Consolidated Innovative Nuclear Research 
    \item[]American Nuclear Society Student Conference -  University of Pittsburgh 
        \vspace{.05in}
    \item[]\textit{2016}
    \item[]American Nuclear Society Idaho Section - Idaho High School Essay Contest 
    \item[]USDOE - SBIR/STTR Phase I Release 2 
    \item[]FY17 TI Portfolio, Bonneville Power Administration, Office of Technology Innovation 
    \item[]American Nuclear Society Student Conference - University of Wisconsin-Madison 
        \vspace{.05in}
    \item[]\textit{2015}
    \item[]American Nuclear Society Idaho Section - Idaho High School Essay Contest
    \item[]FY16 TI Portfolio, Bonneville Power Administration, Office of Technology Innovation 
    \item[]USDOE - Advanced Research Projects Agency-Energy Concept Papers, Open 2015 FOA 
        \vspace{.05in}
    \item[]\textit{2012}
    \item[]USDOE - Advanced Research Projects Agency-Energy Review Panel - Arlington, Virginia
        \vspace{.05in}
    \item[]\textit{2011}
    \item[]International Conf., High-Level Radioactive Waste Management - Albuquerque, New Mexico
        \vspace{.05in}
    \item[]\textit{2010}
    \item[]Clays in Natural \& Engineered Barriers for Radioactive Waste Confinement - Nantes, France
        \vspace{.05in}
    \item[]\textit{2007}
    \item[]Clays in Natural \& Engineered Barriers for Radioactive Waste Confinement - Lille, France 
\end{itemize}

\subsection*{Conference Committees}
\begin{itemize}[leftmargin=*,topsep=0pt,itemsep=.05ex,partopsep=1ex,parsep=1ex]
    \item[]Co-Chair, Technical Program Committee, 2018 Advances in Nuclear Nonproliferation Technology and Policy Conference, American Nuclear Society, 11-15 November, 2018, Orlando, Florida
    \item[]Technical program committee, Advances in Nuclear Nonproliferation Technology and Policy Conference: Bridging the Gaps in Nuclear Nonproliferation, 25-30 September, 2016, Santa Fe, New Mexico
    \item[]Technical program paper review committee, Advances in Nuclear Nonproliferation Technology and Policy Conference: Bridging the Gaps in Nuclear Nonproliferation, 25-30 September, 2016, Santa Fe, New Mexico
\end{itemize}

\subsection*{Technical Session Organizer}
\begin{itemize}[leftmargin=*,topsep=0pt,itemsep=.05ex,partopsep=1ex,parsep=1ex]
    \item[]Cybersecurity for nuclear installations, American Nuclear Society Winter Meeting, 17-21 November, 2019, Washington, D. C. \textit{(co-Organizer with Prof. Jamie B. Coble, University of Tennessee-Knoxville)}
\end{itemize}

\subsection*{Technical Session Chair}
\begin{itemize}[leftmargin=*,topsep=0pt,itemsep=.05ex,partopsep=1ex,parsep=1ex]
    \item[]Spent fuel storage and transportation, American Nuclear Society Winter Meeting, 17-21 November, 2019, Washington, D. C.
    \item[]Cybersecurity for nuclear installations, American Nuclear Society Winter Meeting, 17-21 November, 2019, Washington, D. C. 
    \item[]Data synthesis for pyroprocessing safeguards, 2018 Advances in Nuclear Nonproliferation Technology and Policy Conference (ANTPC), American Nuclear Society, 11-15 November, 2018, Orlando Florida
    \item[]Special session - Prof. Joonhong Ahn Memorial, International High-Level Radioactive Waste Management Conference, 09-13 April, 2017, Charlotte, North Carolina
    \item[]Nonproliferation policy, concepts, and approaches: Treaty verification regimes, state-level concepts and fuel cycle analysis, Advances in Nuclear Nonproliferation Technology and Policy Conference: Bridging the Gaps in Nuclear Nonproliferation, 25-30 September, 2016, Santa Fe, New Mexico
    \item[]Security, safeguards, and non-proliferation, International Conf., High-Level Radioactive Waste Management, 10-14 April 2011, Albuquerque, New Mexico
    \item[]Used Fuel recycling technologies-I, International Congress on Advances in Nuclear Power Plants (ICAPP ‘10) 13-17 June, 2010, San Diego, California
    \item[]Engineered systems and transport processes, International Conf., High-Level Radioactive Waste Management, 07-11 September, 2008, Las Vegas, Nevada
    \item[]Challenges in nuclear waste disposal: Sociological aspects and technical approaches, Global Center of Excellence, 06 December, 2007, The University of Tokyo, Tokyo Japan
\end{itemize}

\subsection*{Symposia Organizer}
\begin{itemize}[leftmargin=*,topsep=0pt,itemsep=.05ex,partopsep=1ex,parsep=1ex]
    \item[]Nuclear cybersecurity research initiatives annual meeting, Center for Advanced Energy Studies, 27-28 July, 2017, Idaho Falls, Idaho
    \item[]Nuclear cybersecurity research focus area identification annual meeting, Center for Advanced Energy Studies, 21-22 July, 2016, Idaho Falls, Idaho
    \item[]Technical implications of nuclear energy system options, Symposium on scientific and institutional aspects of advanced systems for spent nuclear fuels in emerging nuclear countries, Center for International Security and Cooperation, 29-30 September, 2011, Stanford University, Palo Alto, California
\end{itemize}

\subsection*{Invited Talks}
\begin{itemize}[leftmargin=*,topsep=0pt,itemsep=-1ex,partopsep=1ex,parsep=1ex]
    \item[]\textit{University of Idaho $\sq$ Idaho Falls Center for Higher Education}
    \item[]Strategies and success for ethical research $\sq$ Vandal Advantage Graduate Student Orientation \hfill{\textit{August 2019; 2018}}
        \vspace{.05in}
    \item[]\textit{Seminars}
    \item[]University of Tennessee-Knoxville Department of Nuclear Engineering \hfill{\textit{October 2019}}
        \vspace{.05in}
    \item[]\textit{Lightning Talks}
    \item[]Idaho National Laboratory Nuclear Science \& Technology Research Planning Meeting \hfill{\textit{February 2018}}
    \item[]Idaho National Laboratory Energy \& Environment Research Planning Meeting \hfill{\textit{November 2017}}
        \vspace{.05in}
    \item[]\textit{Outreach}
    \item[]P3/TRIO Upward Bound STEM Day \hfill{\textit{December 2015}}
        \vspace{.05in}
    \item[]\textit{Lectures}
    \item[]Stanford University Pre-Collegiate Summer Institutes \hfill{\textit{July 2014}}
\end{itemize}

\subsection*{Community Service}
\begin{itemize}[leftmargin=*,topsep=0pt,itemsep=-1ex,partopsep=1ex,parsep=1ex]
    \item[]\textit{College of Eastern Idaho Machine Tool Technology Advisory Board} \hfill{\textit{2018}}
        \vspace{.05in}
    \item[]\textit{Idaho American Nuclear Society Smoke Detector Donation Program} \hfill{\textit{2016 - }}
    \item[]Administration and delivery of smoke detectors to counties in need across the State of Idaho. Over the past fourteen years, we have worked with more than 70 fire departments to donate more than 5700 smoke detectors to Idaho residents. From 2016-19, we donated nearly 1800 smoke detectors to Arimo, Ashton, Bancroft, Bear Lake, Clear Creek, Declo, Downey, Grace, Hamer, Lava Hot Springs, Roberts, Shelley, and Soda Springs, across southeastern Idaho; Centerville, Clear Creek, Horseshoe Bend, Idaho City, Lowman, Placerville and Valley of the Pines in Boise County; and north to the panhandle in Coeur D'Alene, Mullan, Shoshone, North Side, and West Pend.
        \vspace{.05in}
    \item[]\textit{Idaho American Nuclear Society Highway Cleanup} \hfill{\textit{2016 - }}
    \item[]Garbage cleanup of Miles 122-124 on Interstate 15 twice per year
\end{itemize}

\subsection*{Outreach and Related Professional Activities}
\begin{itemize}[leftmargin=*,topsep=0pt,itemsep=-1ex,partopsep=1ex,parsep=1ex]
    \item[]Utah State University Graduate School Fair \hfill{\textit{September 2019}}
    \item[]Montana Tech Career Fair    \hfill{\textit{September 2019; March 2018}}
    \item[]American Nuclear Society Diversity Social UI ANS Student Sponsorship \hfill{\textit{June 2018; November 2017; June 2017}}
    \item[]University of Idaho Moscow Campus Recruiting \hfill{\textit{January 2020; October 2018; March 2017}}
    \item[]Center for Advanced Energy Studies Seminar Series \hfill{\textit{October 2015-17}}
    \item[]Department of Energy consent based siting meeting \hfill{\textit{July 2016}}
    \item[]PHYSOR University of Idaho Sponsorship \hfill{\textit{April 2016}}
    \item[]Boise State University Nuclear Research Summit \hfill{\textit{March 2016}}
    \item[]UI \& BYU American Nuclear Society Student Social \hfill{\textit{October 2018; February 2016}}
    \item[]BYU-Idaho Career Fair \hfill{\textit{October 2015}}
\end{itemize}

\subsection*{Professional Training}
\noindent\textbf{Los Angeles}\\
\textbf{Participant}\\
\textit{National Science Foundation Grant Development Conference} \hfill{\textit{May 2019}}\\
Key officials representing each NSF program directorate, administrative office, Office of General Counsel, and Office of the Inspector General will participate in this two-day conference. The conference is considered a must, particularly for new faculty, researchers, educators and administrators who want to gain insight into a wide range of important and timely issues at NSF, including: the state of current funding; the proposal and award process; and current and recently updated policies and
procedures.
\vspace*{.75\baselineskip}

\noindent\textbf{University of Illinois Urbana-Champaign}\\
\textbf{Invited Participant}\\
\textit{Collaborative Open Source Curriculum Development Workshop} \hfill{\textit{July 2018}}\\
This workshop concluded a year of faculty interaction at six universities to develop curricula for courses common across the same disciplines at multiple universities in order to reduce the amount of time that any individual professor spends on creating what is essentially duplicate materials. The method proposed in this workshop is based on open source software development, where code is shared in online repositories, reviewed by peers, and contributed to the main project.
\vspace*{.75\baselineskip}

\noindent\textbf{Generation Atomic + GAIN}\\
\textbf{Facilitator}\\
\textit{Nuclear Advocacy and Communications Training} \hfill{\textit{June 2018}}\\
Opposition groups claim nuclear power plants are unsafe. Recently, the U.S. nuclear power industry has been characterized as too expensive and dangerous when compared to other energy sources. As members of the nuclear community, we know that the success of nuclear energy has never been more important to ensuring a positive future for the world – but what can we do to make a difference? This workshop will leave participants energized to tell today’s nuclear power story and be
well-equipped with the tools to do so. Convincing others about the benefits of nuclear involves more than just laying out the facts. Thoughtful and personal storytelling bridges gaps when speaking those who are unfamiliar with the technology by explaining the personal and moral reasons that we work in this field. Telling our personal stories and motivations for working in nuclear creates common ground from which we can better explain nuclear’s benefits: whether it’s as a mother talking to a
father, a surfer talking to a skier, or a cook talking to a conservationist, the human stories that nuclear makes possible are our strongest tools when speaking to the public. The most effective nuclear advocacy takes place at the interpersonal level when we strike up conversations with peers and even better, strangers. Because you can never know who it might be sitting across from you at that dinner party or next to you on the airplane, it’s important to practice having open, considerate
conversations with people of all backgrounds. 
\vspace*{.75\baselineskip}

\noindent\textbf{Pacific Northwest National Laboratory}\\
\textbf{Invited Participant}\\
\textit{Cyber Security Course for Safeguards Practitioners} \hfill{\textit{April 2018}}\\
The course is designed for early to mid-career safeguards practitioners (technical instrument developers, instrument users, policy advisors, etc.) who will benefit from a greater understanding of cyber security threats and how to reduce risks to safeguards systems and processes. This 3.5 day course is designed to teach safeguards and cybersecurity experts how to recognize and mitigate potential cybersecurity vulnerabilities in emerging safeguards instrumentation, information systems, and conduct of operations. This training features classroom style learning opportunities through hands-on exercises and provides plenty of time for questions and discussion. Participants will learn new cyber security skills, use cyber security tools, and collaborate with one another and cyber experts to resolve challenges. They will gain an understanding of cyber security principles and a better awareness of cyber risks associated with safeguards systems. Exercises will include puzzles, exploits of attended and unattended monitoring systems, blended physical and cyber attacks on fictional nuclear facilities, and network defense.
\vspace*{.75\baselineskip}

\noindent\textbf{Idaho National Laboratory Nuclear Science and Technology}\\
\textbf{Invited Participant}\\
\textit{Collaborative Research Planning Meeting} \hfill{\textit{February 2018}}\\
The Center for Advanced Energy Studies (CAES) is a catalyst for collaborative projects focused on energy research, achieved via connections between the CAES entities – Idaho National Laboratory (INL), Boise State University, Idaho State University, University of Idaho, and University of Wyoming – and beyond. Through a series of strategic planning meetings, CAES leadership aims to develop a set of focused research directions. The meeting is intended to establish new collaboration, along with a list of prioritized goals and actions items that will steer internal research investments with the intent of growing sustainable, externally-funded programs. The planning meeting will focus on a strategic area tied to one or more INL directorates and will bring together the appropriate INL and university leadership and researchers.
\vspace*{.75\baselineskip}

\noindent\textbf{Idaho National Laboratory National and Homeland Security Sci \& Tech}\\
\textbf{Invited Participant}\\
\textit{Collaborative Research Planning Meeting} \hfill{\textit{February 2018}}\\
The Center for Advanced Energy Studies (CAES) is a catalyst for collaborative projects focused on energy research, achieved via connections between the CAES entities – Idaho National Laboratory (INL), Boise State University, Idaho State University, University of Idaho, and University of Wyoming – and beyond. Through a series of strategic planning meetings, CAES leadership aims to develop a set of focused research directions. The meeting is intended to establish new collaboration, along with a list of prioritized goals and actions items that will steer internal research investments with the intent of growing sustainable, externally-funded programs. The planning meeting will focus on a strategic area tied to one or more INL directorates and will bring together the appropriate INL and university leadership and researchers.
\vspace*{.75\baselineskip}

\noindent\textbf{Idaho National Laboratory Energy \& Environment Science and Technology}\\
\textbf{Invited Participant}\\
\textit{Collaborative Research Planning Meeting} \hfill{\textit{November 2017}}\\
The Center for Advanced Energy Studies (CAES) is a catalyst for collaborative projects focused on energy research, achieved via connections between the CAES entities – Idaho National Laboratory (INL), Boise State University, Idaho State University, University of Idaho, and University of Wyoming – and beyond. Through a series of strategic planning meetings, CAES leadership aims to develop a set of focused research directions. The meeting is intended to establish new collaboration, along with a list of prioritized goals and actions items that will steer internal research investments with the intent of growing sustainable, externally-funded programs. The planning meeting will focus on a strategic area tied to one or more INL directorates and will bring together the appropriate INL and university leadership and researchers.
\vspace*{.75\baselineskip}

\noindent\textbf{Idaho Falls Post Register}\\
\textbf{Participant}\\
\textit{Intermountain Energy Summit} \hfill{\textit{August 2017; 2016; 2015}}\\
This summit is held annually and covers energy issues unique to the intermountain region. Participants include faculty from local universities, researchers from national laboratories, energy companies, and politicians. This year, the theme is energy security with a focus on nuclear, renewable, and alternative energy sources and continued advancements in grid and cybersecurity.
\vspace*{.75\baselineskip}

\noindent\textbf{Idaho National Laboratory}\\
\textbf{Project Mentor}\\
\textit{Modeling, Experimentation, Validation (MeV) Summer School} \hfill{\textit{July 2017}}\\
The MeV Summer School provides enhanced training for engineers and applied scientists involved in design, licensing, and operation of current and advanced nuclear reactor systems through a multi-faceted learning approach of lectures, tours, and hands-on activities. The school is being organized through the cooperation of national laboratories, industry, government agencies, and universities that share the goal of building a strong workforce to support global nuclear expansion. The faculty will be drawn from the top experts in academia, industry, and government. The general organization and conduct of the school will be overseen by an international board of senior experts. A local secretariat will provide technical, logistical and administrative support to students and faculty. It is the aim of the school to foster the development of a next-generation network of scientists and engineers capable of advancing nuclear energy in the 21st century through integrated modeling and experimentation.
\vspace*{.75\baselineskip}

\noindent\textbf{University of Illinois Urbana-Champaign}\\
\textbf{Invited participant}\\
\textit{PyNE Summit} \hfill{\textit{June 2017}}\\
PyNE is a suite of tools to aid in computational nuclear science and engineering. PyNE seeks to provide native implementations of common nuclear algorithms, as well as Python bindings and I/O support for other industry standard nuclear codes.
\vspace*{.75\baselineskip}

\noindent\textbf{University of Illinois Urbana-Champaign}\\
\textbf{Invited Participant}\\
\textit{Collaborative Open Source Curriculum Development Workshop} \hfill{\textit{June 2017}}\\
This workshop included faculty at six universities to develop curricula for courses common across the same disciplines at multiple universities in order to reduce the amount of time that any individual professor spends on creating what is essentially duplicate materials. The method proposed in this workshop is based on open source software development, where code is shared in online repositories, reviewed by peers, and contributed to the main project.
\vspace*{.75\baselineskip}

\noindent\textbf{Pacific Northwest National Laboratory}\\
\textbf{Invited Participant}\\
\textit{Pacific Northwest National Laboratory Safeguards Laboratory Day} \hfill{\textit{May 2019; 2018; 2017}}\\
Students and faculty from the University of Idaho were invited to Pacific Northwest National Laboratory to learn about the research activities at the laboratory. The day also included hands-on safeguards and security experiments conducted at laboratory facilities, such as materials accounting and vehicle searches.
\vspace*{.75\baselineskip}

\noindent\textbf{Department of Energy}\\
\textbf{Participant}\\
\textit{Consent-Based Siting Public Meeting} \hfill{\textit{July 2016}}\\
The Department of Energy is in the initial phase of developing a consent-based process for siting the facilities needed to store and dispose of the nation’s spent nuclear fuel and high-level radioactive waste.  A consent-based approach to siting relies on understanding the views of the public, stakeholders, and governments at the local, state, and tribal levels.  In this first phase, the Department is engaging with interested groups and individuals to learn about what elements are important to consider in designing an enduring approach to siting.  This session is an opportunity for the public to share thoughts and perspectives through a facilitated discussion.
\vspace*{.75\baselineskip}

\noindent\textbf{Global American Business Institute}\\
\textbf{Invited Participant}\\
\textit{Trilateral Nuclear Energy Dialogue: Korea, Japan, United States} \hfill{\textit{July 2016}}\\
Convene a high-level private workshop among preeminent Korean, Japanese, and American experts in nuclear energy and nuclear policy issues, with the intention of fostering relationships, confidence building, and seeking potential areas for trilateral cooperation. In keeping with the overarching theme of previous discussions—the back-end fuel cycle—this meeting seeks to underscore the role of advanced nuclear energy and fuel cycle technologies. Although the obstacles impeding permanent solutions to spent fuel and radioactive waste are largely political, this dialogue aims to highlight the potential of cutting-edge technologies in addressing the policy, environmental, and public acceptance challenges facing management of the nuclear fuel cycle, in addition to opportunities for international collaboration in the research, development, demonstration, and deployment of such technologies.
\vspace*{.75\baselineskip}

\noindent\textbf{Pacific Northwest National Laboratory}\\
\textbf{Invited Participant}\\
\textit{Next Generation Safeguards Initiative Summer Course} \hfill{\textit{June 2016}}\\
This course, offered through the DOE/NNSA Next Generation Safeguards Initiative (NGSI), covers major international safeguards procedures currently in use in IAEA member nations. Daily lectures were supplemented with hands-on safeguards activities conducted by IAEA safeguards inspectors and researchers at PNNL. Participants included faculty, postdoctorate researchers, and graduate students.
\vspace*{.75\baselineskip}

\noindent\textbf{Boise State University}\\
\textbf{Participant}\\
\textit{Idaho's Role in Nuclear: Clean Energy Powered by the Next Generation} \hfill{\textit{March 2016}}\\
Boise State University is proud to bring globally recognized leaders in nuclear energy to address the benefits and challenges associated with nuclear energy production and its role in supplying clean energy for a growing world. The summit will also include a panel discussion with John Kotek, Assistant Director of Nuclear Energy, U.S. Department of Energy; Dr. Mark Peters, Laboratory Director for the Idaho National Laboratory; Mark Rudin, Vice President for Research at Boise State University; Mike McGough, Chief Commercial Officer at NuScale Power, and Dr. Harold Blackman, Associate Vice President for Research and Economic Development at Boise State University. The panel will address questions about the benefits of and concerns around nuclear power.
\vspace*{.75\baselineskip}

\noindent\textbf{Portland State University}\\
\textbf{Participant}\\
\textit{National Science Foundation Grant Development Conference} \hfill{\textit{February 2016}}\\
Key officials representing each NSF program directorate, administrative office, Office of General Counsel, and Office of the Inspector General will participate in this two-day conference. The conference is considered a must, particularly for new faculty, researchers, educators and administrators who want to gain insight into a wide range of important and timely issues at NSF, including: the state of current funding; the proposal and award process; and current and recently updated policies and procedures.
\vspace*{.75\baselineskip}

\noindent\textbf{University of California-Berkeley}\\
\textbf{Invited Participant}\\
\textit{Proliferation Resistance and Physical Protection Working Group Workshop} \hfill{\textit{November 2015}}\\
The PR\&PP methodology was developed within the Generation IV International Forum (GIF) to provide a structured framework to assess the proliferation resistance and physical protection robustness of Gen IV nuclear energy systems, and to guide designers to further improve their systems. This workshop is intended to familiarize non-experts in this field with the broad aspects of the methodology and its applications. The PR\&PP Working Group will present an overview of the methodology to an audience of  students, academics, and members of the GIF community who wish to become more familiar with the PR\&PP methodology. To illustrate the methodology, its application to a hypothetical nuclear energy system will be examined. Workshop participants will be divided into subgroups that will consider different proliferation and security threats, and will identify important design features and approaches that contribute to the system’s resilience to these threats. Following these subgroup sessions, workshop participants will reconvene to review insights from the subgroups. Finally, an open discussion will be held to obtain feedback from the participants on the GIF approach to PR\&PP with the objective of refining the methodology and its presentation to the wider community of academics and prospective GIF users.
\vspace*{.75\baselineskip}

\noindent\textbf{State of Idaho Board of Education $\sq$ Boise, Idaho}\\
\textbf{Invited participant}\\
\textit{Open Education Resources Development} \hfill{\textit{October 2015}}\\
This workshop focused on the use of open source educational materials in order to produce an online textbook as a supplement to existing commercial textbooks. An online textbook allows flexibility for faculty to augment content without requiring multiple textbooks for a course. An online textbook allows the educational content of a course to be more closely aligned with the desired learning outcomes. Two online texts have been developed: Principles of Nuclear Engineering and Risk Assessment.
\vspace*{.75\baselineskip}

\noindent\textbf{University of California-Berkeley $\sq$ Department of Nuclear Engineering}\\
\textbf{Discussant}\\
\textit{Advanced Summer School of Radioactive Waste Disposal with Social-Scientific Literacy} \hfill{\textit{August 2011}}\\
\textit{Reflections on the Fukushima Nuclear Accident and Beyond}\\
This advanced summer school was established in conjunction with the Department of Nuclear Engineering/Management, the University of Tokyo and the Department of Nuclear Engineering, University of California, Berkeley to provide Ph.D. students and early-career nuclear engineers with education in social sciences and engineering. The goal was to foster a next generation of engineers capable of understanding the public and societal needs, contributing to the societal decision making, and taking a responsible role as engineering experts in society. The discussant leads student group activities by stimulating questions from students and corroborating with the chair to develop a summary of lectures.
\vspace*{.75\baselineskip}

\noindent\textbf{University of California-Berkeley $\sq$ Department of Nuclear Engineering}\\
\textbf{Invited Participant}\\
\textit{Minner Fellows Program} \hfill{\textit{June 2011}}\\
The objective of this program was to develop a framework for making ethical judgments in engineering. Because engineering faculty and graduate students play a leadership role in the development of these technologies, it is essential that they become aware the ethical, legal and social ramifications of them. The course focused on context, or the embodiment of moral maturity and ethical expertise, in the same way that faculty and graduate students embody engineering and technical expertise.
\vspace*{.75\baselineskip}

\noindent\textbf{Hawai'i Tokai International College}\\
\textbf{Project Mentor}\\
\textit{Advanced Summer School of Radioactive Waste Disposal with Social-Scientific Literacy} \hfill{\textit{August 2010}}\\
Special emphasis was placed on integrating nuclear science and engineering with social science. A series of lectures and student group discussions were conducted, followed by a student workshop in which projects were developed, under the theme of ‘nuclear engineers in society,’ including topics such as the safety of a high level waste repository and nuclear systems for non-nuclear weapons states. Project mentors provided guidance, direction, and oversight. The school ended with a series of project presentations and submittal of project reports.
\vspace*{.75\baselineskip}

\noindent\textbf{University of California-Berkeley $\sq$ Department of Nuclear Engineering}\\
\textbf{Student}\\
\textit{Advanced Summer School of Radioactive Waste Disposal with Social-Scientific Literacy} \hfill{\textit{August 2009}}\\
In this summer school, The integration of nuclear science and engineering with social science was emphasized through a series of lectures, panel and group discussions. Students directed discussions based on the activities each day. The school culminated in a tour of the Waste Isolation Pilot Plant (WIPP).
\vspace*{.75\baselineskip}

\noindent\textbf{University of California-Berkeley $\sq$ Graduate Student Instructor Teaching \& Resource Center}\\
\textbf{Participant}\\
\textit{Summer Institute for Preparing Future Faculty} \hfill{\textit{May - June 2005}}\\
This unique program is for Doctoral Candidates with an interest in an academic career. Many aspects of teaching are covered: course design, syllabus development, teaching and learning assessment, teaching and learning strategies, and the creation of a teaching portfolio. The program exposes candidates to faculty in several disciplines both within and outside the university; thus allowing for the dissemination of the full scope of teaching methods and skills, as well as broadening of perspectives with regards to the entire teaching profession.
\vspace*{.75\baselineskip}

\subsection*{Awards and Fellowships}
\begin{itemize}[leftmargin=*,topsep=0pt,itemsep=-1ex,partopsep=1ex,parsep=1ex]
    \item[]\textbf{University of Idaho}
    \item[]\textit{American Nuclear Society University of Idaho Student Section}
    \item[]Samuel Glasstone Award for Public Service $\sq$ Second Place \hfill{\textit{2019}}
    \item[]Certificate of Distinction \hfill{\textit{2019}}
        \vspace{.05in}
    \item[]\textbf{University of California-Berkeley}
    \item[]Nuclear Engineering Department Block Grant Fellowship \hfill{\textit{2005}}
    \item[]Outstanding Graduate Student Instructor Award \hfill{\textit{2003}}
    \item[]Katherina S. DeSharton Fellowship \hfill{\textit{1999}}
    \item[]Hamilton Family Memorial Fellowship \hfill{\textit{1999}}
        \vspace{.05in}
    \item[]\textbf{United States Department of Energy $\sq$ Office of Civilian Radioactive Waste Management}
    \item[]Civilian Radioactive Waste Management Fellowship \hfill{\textit{2000 - 2004}}
        \vspace{.05in}
    \item[]\textbf{Nuclear Energy Institute}
    \item[]National Academy for Nuclear Training Fellowship \hfill{\textit{1999}}
\end{itemize}

\subsection*{Professional Societies and Positions}
\begin{itemize}[leftmargin=*,topsep=0pt,itemsep=-1ex,partopsep=1ex,parsep=1ex]
    \item[]\textbf{American Nuclear Society}
    \item[]Executive Committee $\sq$ Nonproliferation Policy Division \hfill{\textit{2019 - 2022}}
    \item[]Executive Committee $\sq$ Student Sections Committee \hfill{\textit{2018 - 2021}}
    \item[]Executive Committee $\sq$ Fuel Cycle and Waste Management Division, American Nuclear Society \hfill{\textit{2018 - 2021}}
        \vspace{.05in}
    \item[]\textbf{Idaho Section of the American Nuclear Society}
    \item[]Community Service \hfill{\textit{2016 - }}
    \item[]Board of Directors \hfill{\textit{2018 - 2019}}
        \vspace{.05in}
    \item[]Tau Beta Pi Engineering Society
\end{itemize}

\end{document}
