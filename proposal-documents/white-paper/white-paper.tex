%@TheDoctorRAB
%Standard white paper/preproposal format when there is a page limit
%neup.bst designed to use numbered citations in order of appearance, short author list
%nsf.st is for numbers, full author list
%standard.bst is inline citations with author last name
\documentclass[11pt,letterpaper]{article}
\usepackage[lmargin=1in,rmargin=1in,tmargin=1in,bmargin=1in]{geometry}
\usepackage[pagewise]{lineno} %line numbering
\usepackage{setspace}
\usepackage{ulem} %strikethrough - do not \sout{\cite{}}
\usepackage{xcolor,colortbl} %change font color
\usepackage{graphicx}
\usepackage{filecontents}
\usepackage{tablefootnote}
\usepackage{footnotehyper}
%\usepackage{subfig}
\usepackage[yyyymmdd]{datetime} %date format
\renewcommand{\dateseparator}{.}
\graphicspath{{../img/}} %path to graphics
\setcounter{secnumdepth}{5} %set subsection to nth level
\usepackage{times}
\usepackage{tabto} %general tabbed spacing
\usepackage{longtable} %need to put label at top under caption then \\ - use spacing
\usepackage[stable,hang,flushmargin]{footmisc} %footnotes in section titles and no indent; standard.bst
%\usepackage[round,semicolon]{natbib} %use 'numbers' for numbered citations; 'round' for () instead [] for inline citations
\usepackage[numbers,sort&compress]{natbib} %use 'numbers' for numbered citations; 'round' for () instead [] for inline citations; nsf.bst, neup.bst
\usepackage{enumitem}
\usepackage{boldline}
\usepackage{makecell}
\usepackage{booktabs}
\usepackage{amssymb}
\usepackage{amsmath}
\usepackage{physics}
\usepackage{tabularx}
\usepackage{multirow}
\usepackage{lscape}
\usepackage{array}
\usepackage{caption}
\usepackage{subcaption}
\usepackage[labelfont=bf]{caption}
\usepackage{chngcntr}
%\usepackage{hyperref}
\usepackage{sectsty}
\usepackage{pdfpages}
\usepackage{textcomp}
\usepackage{lastpage}
\usepackage[toc,title]{appendix}
\usepackage[figure,table]{totalcount}
\usepackage[acronym,nomain,nonumberlist]{glossaries}
\makenoidxglossaries

\usepackage[singlelinecheck=false]{caption}
\captionsetup[table]{skip=7pt} %sets a space after table caption
\captionsetup[figure]{skip=7pt,labelformat={default},labelsep=period} %sets space above caption, 'figure' format

\usepackage{wrapfig}
\setlength{\intextsep}{0.20mm}
\setlength{\columnsep}{0.20mm}

%\usepackage{xr} %for revisions - will cross reference from one file to here
%\externaldocument{/path/to/auxfilename} %aux file needed

\newcommand{\edit}[1]{\textcolor{blue}{#1}} %shortcut for changing font color on revised text
\newcommand{\fn}[1]{\footnote{#1}} %shortcut for footnote tag
\newcommand*\sq{\mathbin{\vcenter{\hbox{\rule{.3ex}{.3ex}}}}} %makes a small square as a separator $\sq$
\newcommand{\sk}[1]{\sout{#1}} %shortcut for strikethrough
\newcommand{\x}{\cellcolor{lightgray}} %use to shade in table cell
\newcommand{\acf}{\acrfull} %full acronym
\newcommand{\acl}{\acrlong} %long acronym
\newcommand{\acs}{\acrshort} %short acronym

\newcolumntype{L}[1]{>{\raggedright\let\newline\\\arraybackslash\hspace{0pt}}p{#1}} %uses \raggedright with p{} in table column

\makeatletter
\renewcommand\tableofcontents{%
    \@starttoc{toc}%
}
\makeatother

\makeatletter
\renewcommand\listoffigures{%
    \@starttoc{lof}%
}
\makeatother

\makeatletter
\renewcommand\listoftables{%
    \@starttoc{lot}%
}
\makeatother

\makeatletter
\renewcommand\section{%
    \@startsection{section}{1}{\z@ }{0.50\baselineskip}{0.25\baselineskip}
    {\normalfont \normalsize \bfseries}}%

\makeatletter
\renewcommand\paragraph{%
    \@startsection{paragraph}{4}{\z@ }{0.35\baselineskip}{-1em}
    {\normalfont \normalsize \bfseries}}%

\makeatletter
\renewcommand\subparagraph{%
    \@startsection{subparagraph}{5}{\z@ }{0.25\baselineskip}{-1em}
    {\normalfont \normalsize \itshape }}%

\makeatletter
\renewcommand\subsection{%
    \@startsection{subsection}{2}{\z@ }{0.45\baselineskip}{0.25\baselineskip}
    { \large \normalsize \bfseries}}%
    
%\sectionfont{\normalsize}
%\subparagraphfont{\normalfont\itshape}

\setlength{\bibsep}{0pt} %sets space between references
\renewcommand{\bibsection}{} %suppresses large 'references' heading
\renewcommand\bibpreamble{\vspace{-0.2\baselineskip}} %sets spacing after heading if not using default references heading

\usepackage{fancyhdr}
\pagestyle{fancy}
\fancyhf{} %move page number to bottom right
\renewcommand{\headrulewidth}{0pt} %set line thickness in header; uncomment as is to remove line
%\lhead{\scriptsize agency}
%\chead{\scriptsize title}
%\rhead{\scriptsize \today}
\rfoot{\thepage}

\begin{filecontents}{references.bib}
\end{filecontents}

%\newacronym{}{}{}

%spacing options
%\onehalfspacing %linespacing
%\setstretch{1.05} %linespacing
%\spacing{1.25} %equivalent to 1.5 line spacing in Word

\begin{document}

{\centering 
    \textbf{Title\\
    Call\\
    }
    Name (PI) - Affiliation\\
    Name (co-PI) - Affiliation 
\par
}

\noindent\makebox[\linewidth]{\rule{\textwidth}{0.5pt}} %horizontal line spanning margins

\paragraph*{Summary of the Proposed Project.} 
\noindent

\paragraph*{Motivation.}
\noindent

\paragraph*{Overall Objective.}
\noindent

\paragraph*{Importance and Relevance to Objectives.} 
\subparagraph*{Background.} 
\noindent

\subparagraph*{Relevant literature.}
\noindent

%for a 3 page preproposal, limit above subject matter to the first page

\paragraph*{Logical Path, Work Scope, Description of Tasks.} The Logical Path and the proposed Work Scope is described below in terms of the defined Tasks for the project.

\vspace{0.25\baselineskip}

\paragraph*{Workscope 1 - Title.} (\textit{Lead - ; Support - })
\subparagraph*{Task I. Title.}
\noindent

\subparagraph*{Task II. Title.}
\noindent

\paragraph*{Workscope 2 - Title.} (\textit{Lead - ; Support - })
\subparagraph*{Task II. Title.}
\noindent

\subparagraph*{Task IV. Title.}
\noindent

\paragraph*{Logical Path to Work Accomplishment. Major Deliverables and Outcomes.} 
\noindent We anticipate the following outcomes - 
\begin{enumerate}[topsep=0pt,itemsep=-0.75ex,partopsep=1ex,parsep=1ex]
    \item
    \item
    \item
    \item Identify lessons learned and engage in future pathways and partnerships.
\end{enumerate}

\paragraph*{Estimated Cost of the Project.} 
\noindent We request a total of \$X,XXX,XXX for this project for a timeframe of N years.

%limiting tasks, deliverables to one page allows for more references to be included on the third page, for a 3 page limit

%\begin{table}[h!]
%    \centering
%    \caption*{\textbf{Timeframe for Execution of Proposed Project. Schedule, Roles, and Responsibilities.}}
%    \begin{tabular}{|L{0.17\linewidth}|c|c|c|c|c|c|c|c|c|c|c|c|}
\begin{longtable}{|L{0.17\linewidth}|c|c|c|c|c|c|c|c|c|c|c|c|}
        \caption*{\textbf{Timeframe for Execution of Proposed Project. Schedule, Roles, and Responsibilities.}}
        \label{tab-timeline}\\
        \hline
        \multicolumn{1}{|c|}{\multirow{2}{*}{\textbf{TASKS}}}& 
        \multicolumn{4}{|c|}{\textbf{Y1}}&
        \multicolumn{4}{|c|}{\textbf{Y2}}&
        \multicolumn{4}{|c|}{\textbf{Y3}}
        \\
        \cline{2-13}
        &
        \textbf{Q1}& 
        \textbf{Q2}& 
        \textbf{Q3}& 
        \textbf{Q4}& 
        \textbf{Q1}& 
        \textbf{Q2}& 
        \textbf{Q3}& 
        \textbf{Q4}& 
        \textbf{Q1}& 
        \textbf{Q2}& 
        \textbf{Q3}& 
        \textbf{Q4}
        \\
        \hline
        I. 
        &\x\textbf{I}
        & 
        & 
        & 
        & 
        & 
        & 
        & 
        & 
        & 
        & 
        & 
        \\
        \hline
        II. 
        &\x\textbf{II} 
        & 
        & 
        & 
        & 
        & 
        & 
        & 
        & 
        & 
        & 
        & 
        \\
        \hline
        III. 
        &\x\textbf{III} 
        & 
        & 
        & 
        & 
        & 
        & 
        & 
        & 
        & 
        & 
        & 
        \\
        \hline
        IV. 
        &\x\textbf{IV}
        & 
        & 
        & 
        & 
        & 
        & 
        & 
        & 
        & 
        & 
        & 
        \\
        \hline
        V. 
        &\x\textbf{V} 
        & 
        & 
        & 
        & 
        & 
        & 
        & 
        & 
        & 
        & 
        & 
        \\
        \hline
        VI. 
        &\x\textbf{VI} 
        & 
        & 
        & 
        & 
        & 
        & 
        & 
        & 
        & 
        & 
        & 
        \\
        \hline
        VII. Crosscutting issues
        & 
        & 
        & 
        & 
        & 
        & 
        & 
        & 
        & 
        & 
        &\x\textbf{VII} 
        &\x\textbf{VII} 
        \\
        \hline
%    \end{tabular}
%    \label{tab-timeframe}
%\end{table}
\end{longtable}

\newpage
%remove page break if there are page limits
%\vspace{0.5\baselineskip}

\noindent\textbf{References}
\bibliographystyle{neup}
\setlength{\bibhang}{0pt}
\bibliography{references}

%wrap figure around text
%move to wherever in the text 
%{l} fixes it on the left margin
%{L} floats
%\begin{wrapfigure}{l}{0.x\textwidth}
%       \includegraphics[width=0.(x-2)\textwidth]{}
%    \caption{}
%    \label{fig-}
%\end{wrapfigure}

\end{document}
