%@TheDoctorRAB
%standard white paper/preproposal format
%
%%%%%
%
%REFERENCES
%
%neup.bst - numbered citations in order of appearance, short author list with et al in reference section
%nsf.bst - numbered citations in order of appearance, full author list in references section
%standard.bst - citations with author last name with et al for more than 2 authors; full author list in references section
%ans.bst is for ANS only. 
%
%author = {Lastname, Firstname and Lastname, Firstname and Lastname, Firstname} for all bst formats
%bst renders the author list itself
%
%author = {{Nuclear Regulatory Commission}} if the author is an organization, institution, etc., and not people
%
%title = {{}} for all
%
%for all - use \citep{-} - [1] or (Borrelli, 2021) in the text
%standard.bst \cite{-} - Borrelli (2021) in the text
%standard.bst lists references alphabetically
%the rest list numerically
%
%%%%%
\documentclass[11pt,letterpaper]{article}
\usepackage[lmargin=1in,rmargin=1in,tmargin=1in,bmargin=1in]{geometry}
\usepackage[pagewise]{lineno} %line numbering
\usepackage{setspace}
\usepackage{ulem} %strikethrough - do not \sout{\cite{}}
\usepackage[pdftex,dvipsnames]{xcolor,colortbl} %change font color
\usepackage{graphicx}
\usepackage{filecontents}
\usepackage{tablefootnote}
\usepackage{footnotehyper}
\usepackage{float}
%\usepackage{subfig}
\usepackage[yyyymmdd]{datetime} %date format
\renewcommand{\dateseparator}{.}
\graphicspath{{../img/}} %path to graphics
\setcounter{secnumdepth}{5} %set subsection to nth level

%%%%% fonts
\usepackage{times}
%arial - uncomment next two lines
%\usepackage{helvet}
%\renewcommand{\familydefault}{\sfdefault}
%%%%%

%%%%% references
%\usepackage[round,semicolon]{natbib} %for (Borrelli 2021; Clooney 2019) - standard.bst 
\usepackage[numbers,sort&compress]{natbib} %for [1-3] - nsf.bst, neup.bst
%%%%%

%%%%% tables and figures
\usepackage{longtable} %need to put label at top under caption then \\ - use spacing
\usepackage{tablefootnote}
\usepackage{tabularx}
\usepackage{multirow}
\usepackage{tabto} %general tabbed spacing

\usepackage[singlelinecheck=false,labelfont=bf]{caption}
\usepackage{subcaption}
\captionsetup[table]{skip=7pt,labelformat={default},labelsep=period} %sets a space after table caption
\captionsetup[figure]{skip=7pt,labelformat={default},labelsep=period} %sets space above caption, 'figure' format

\usepackage{wrapfig} %wraps figures around text
\setlength{\intextsep}{0.20mm}
\setlength{\columnsep}{0.20mm}
%%%%%

\usepackage[stable,hang,flushmargin]{footmisc} %footnotes in section titles and no indent; standard.bst
\usepackage[inline]{enumitem}
\usepackage{boldline}
\usepackage{makecell}
\usepackage{booktabs}
\usepackage{amssymb}
\usepackage{gensymb}
\usepackage{amsmath}
\usepackage{physics}
\usepackage{lscape}
\usepackage{array}
\usepackage{chngcntr}
%\usepackage{hyperref}
\usepackage{sectsty}
\usepackage{textcomp}
\usepackage{lastpage}
\usepackage{xargs} %for \newcommandx
\usepackage[colorinlistoftodos,prependcaption,textsize=small]{todonotes} %makes colored boxes for commenting
\usepackage[toc,page]{appendix}
\usepackage[figure,table]{totalcount}
\usepackage[acronym,nomain,nonumberlist]{glossaries}
\makenoidxglossaries

\usepackage[firstpage]{draftwatermark}
\SetWatermarkText{\shortstack{DRAFT\\do not distribute}}
\SetWatermarkScale{0.35}

\usepackage[singlelinecheck=false]{caption}
\captionsetup[table]{skip=7pt,labelformat={default},labelsep=period,name={Tab.}} %sets a space after table caption
\captionsetup[figure]{skip=7pt,labelformat={default},labelsep=period,name={Fig.}} %sets space above caption, 'figure' format

\usepackage{wrapfig}
\setlength{\intextsep}{0.20mm}
\setlength{\columnsep}{0.20mm}

%\usepackage{xr} %for revisions - will cross reference from one file to here
%\externaldocument{/path/to/auxfilename} %aux file needed

\newcommand{\edit}[1]{\textcolor{blue}{#1}} %shortcut for changing font color on revised text
\newcommand{\fn}[1]{\footnote{#1}} %shortcut for footnote tag
\newcommand*\sq{\mathbin{\vcenter{\hbox{\rule{.3ex}{.3ex}}}}} %makes a small square as a separator $\sq$
\newcommand{\sk}[1]{\sout{#1}} %shortcut for strikethrough
\newcommand{\x}{\cellcolor{lightgray}} %use to shade in table cell

\newcommand{\acf}{\acrfull} %full acronym
\newcommand{\acl}{\acrlong} %long acronym
\newcommand{\acs}{\acrshort} %short acronym

\newcommand{\acfp}{\acrfullpl} %full acronym plural
\newcommand{\aclp}{\acrlongpl} %long acronym plural
\newcommand{\acsp}{\acrshortpl} %short acronym plural

\newcommandx{\que}[2][1=]{\todo[linecolor=red,backgroundcolor=red!25,bordercolor=red,#1]{#2}} %query
\newcommandx{\sug}[2][1=]{\todo[linecolor=blue,backgroundcolor=blue!25,bordercolor=blue,#1]{#2}} %suggested change
\newcommandx{\cmt}[2][1=]{\todo[linecolor=OliveGreen,backgroundcolor=OliveGreen!25,bordercolor=OliveGreen,#1]{#2}} %comment
\newcommandx{\omt}[2][1=]{\todo[linecolor=Plum,backgroundcolor=Plum!25,bordercolor=Plum,#1]{#2}} %omit

\newcolumntype{L}[1]{>{\raggedright\let\newline\\\arraybackslash\hspace{0pt}}p{#1}} %uses \raggedright with p{} in table column

\makeatletter
\renewcommand\tableofcontents{%
    \@starttoc{toc}%
}
\makeatother

\makeatletter
\renewcommand\listoffigures{%
    \@starttoc{lof}%
}
\makeatother

\makeatletter
\renewcommand\listoftables{%
    \@starttoc{lot}%
}
\makeatother

\makeatletter
\newcommand*\ftp{\fontsize{16.5}{17.5}\selectfont}
\makeatother

\makeatletter
\renewcommand\section{%
    \@startsection{section}{1}{\z@ }{0.50\baselineskip}{0.25\baselineskip}
    {\Large \normalfont \bfseries}}%

\makeatletter
\renewcommand\paragraph{%
    \@startsection{paragraph}{4}{\z@ }{0.25\baselineskip}{-1em}
    {\normalfont \normalsize \bfseries}}%

\makeatletter
\renewcommand\subparagraph{%
    \@startsection{subparagraph}{5}{\z@ }{0.20\baselineskip}{-1em}
    {\normalfont \normalsize \itshape }}%

\makeatletter
\renewcommand\subsection{%
    \@startsection{subsection}{2}{\z@ }{0.45\baselineskip}{0.25\baselineskip}
    { \large \normalfont \bfseries}}%
    
\setlength{\bibsep}{0pt} %sets space between references
\renewcommand{\bibsection}{} %suppresses large 'references' heading
\renewcommand\bibpreamble{\vspace{\baselineskip}} %sets spacing after heading if not using default references heading

\usepackage{fancyhdr}
\pagestyle{fancy}
\fancyhf{} %move page number to bottom right
%\renewcommand{\headrulewidth}{0pt} %set line thickness in header; uncomment as is to remove line
\lhead{\scriptsize Agency}
\chead{\scriptsize \textit{Title}}
\rhead{\scriptsize \today}
\rfoot{\thepage}

\begin{filecontents}{references.bib} 
    @misc{
        nye15a,
        author = {Nye, Jr., Joseph S.},
        title = {{The World Needs an Arms-control Treaty for Cybersecurity}},
        year = {2015},
        note = {{The Washington Post}}
    }
    @article{
        mil20a,
        author = {Miller, James N.},
        title = {No to no first use — for now},
        journal = {Bulletin of the Atomic Scientists},
        volume = {76},
        pages = {8},
        year  = {2020}
    }
    @incollection{
        mil17a,
        author = {Miller, Steven E.},
        title = {{Cyber Threats, Nuclear Analogies? Divergent Trajectories in Adapting to New Dual-Use Technologies}},
        booktitle = {Understanding Cyber Conflict: 14 Analogies},
        publisher = {Georgetown University Press},
        year = {2017},
        editor = {Perkovich, George and Levite, Ariel E.},
        pages = {161}
    }
    @conference{
        ben20a,
        author = {Benjamin, Jacob and Haney, Michael},
        title = {{Nonproliferation of Cyber Weapons}},
        year = { 2020},
        organization = {Symposium on Cyber Warfare, Cyber Defense, \& Cyber Security },
        address = {Las Vegas, Nevada}
    }
\end{filecontents}

%%%%% acronyms
% alphabetical ordering is automated
\newacronym{nrs}{NRHES}{Nuclear Renewable Hybrid Energy System}
\newacronym{ahp}{AHP}{Analytical Hierarchy Process}
\newacronym{inl}{INL}{Idaho National Laboratory}
\newacronym{orl}{ORNL}{Oak Ridge National Laboratory}
\newacronym{anl}{ANL}{Argonne National Laboratory}
\newacronym{npp}{NPP}{Nuclear Power Plant}
\newacronym{smr}{SMR}{Small Modular Reactor}
\newacronym{ump}{UAMPS}{Utah Associated Municipal Power Systems}
\newacronym{nus}{NuScale}{NuScale Power, LLC}
\newacronym{nrc}{NRC}{United States Nuclear Regulatory Commission}
\newacronym{epri}{EPRI}{Electric Power Research Institute}
\newacronym{nerc}{NERC}{North American Electric Reliability Corporation}
\newacronym{ci}{CI}{Consistency Index}
\newacronym{cr}{CR}{Consistency Ratio}
\newacronym{htse}{HTSE}{High Temperature Steam Electrolysis}
\newacronym{lwr}{LWR}{Light Water Reactor}
\newacronym{eia}{EIA}{U.S. Energy Information Administration}
\newacronym{oer}{OER}{Online Educational Resource}
\newacronym{lms}{LMS}{Learning Management System}
%\newacronym{}{}{}
%%%%%

%spacing options
%\onehalfspacing %linespacing
%\setstretch{1.05} %linespacing
%\spacing{1.25} %equivalent to 1.5 line spacing in Word

%linenumbering
\linenumbers %toggle line numbers
\pagewiselinenumbers %reset line numbers on new page
\modulolinenumbers[1] %line numbering interval

\begin{document}

{\centering 
    \textbf{Agency - Call \\
    Federal POC - , Technical POC - \\
    General budget\\
    }
\par
}

\noindent Text of call.

\newpage

{\centering
    \textbf{Title\\
    Call\\
    }
    Name (PI) - Affiliation; Name (co-PI) - Affiliation\\
    Name (co-PI) - Affiliation
\par
}

\noindent\makebox[\linewidth]{\rule{\textwidth}{0.5pt}} %horizontal line spanning margins

%may not be needed
\begin{center}\subsection*{Abstract}\end{center}

\paragraph*{Summary of the Proposed Project.} 
\subparagraph*{Proposal.}

\subparagraph*{Key features of the project.}

\paragraph*{Motivation.}
\subparagraph*{Context.}

\paragraph*{Overall Objective.}

\subsection*{Research Focus and Workscopes.} 
\begin{enumerate}[topsep=0pt,itemsep=-1ex,partopsep=1ex,parsep=1ex]
    \item\textit{Workscope title (Lead - )}
        \begin{itemize}[leftmargin=*,topsep=-1ex,itemsep=-1ex,partopsep=1ex,parsep=1ex]
            \item Task 
            \item Task 
            \item Task 
            \item Task 
        \end{itemize}
    \item\textit{Concluding work (Full team)}
        \begin{itemize}[leftmargin=*,topsep=-1ex,itemsep=-1ex,partopsep=1ex,parsep=1ex]
            \item Discuss crosscutting issues
        \end{itemize}
\end{enumerate}

\paragraph*{Importance and Relevance to Objectives.} 
\subparagraph*{Background.} 

\subparagraph*{Relevant literature.}

\subparagraph*{Recently funded work.}
%for a 3 page preproposal, limit above subject matter to the first page
%research focus will have to go

\paragraph*{Unique project features.}

\paragraph*{Logical Path, Work Scope, Description of Tasks.} 

\paragraph*{Workscope 1 - Title.} (\textit{Lead - ; Support - })
\begin{enumerate}[topsep=0pt,itemsep=-0.5ex,partopsep=1ex,parsep=1ex,label=\textit{Task \Roman*.},align=left,leftmargin=0.5in]
    \item\textit{Title.}
    \item\textit{Title.}
    \item\textit{Title.}
\end{enumerate}

\paragraph*{Workscope 2 - Title.} (\textit{Lead - ; Support - })
\begin{enumerate}[resume*,topsep=0pt,itemsep=-0.5ex,partopsep=1ex,parsep=1ex,label=\textit{Task \Roman*.},align=left,leftmargin=0.5in]
    \item\textit{Title.}
    \item\textit{Title.}
    \item\textit{Title.}
\end{enumerate}

\paragraph*{Workscope 3 - Future outlook.} (\textit{Full team})
\begin{enumerate}[resume*,topsep=0pt,itemsep=-0.5ex,partopsep=1ex,parsep=1ex,label=\textit{Task \Roman*.},align=left,leftmargin=0.5in]
    \item\textit{Crosscutting issues.}
\end{enumerate}

\paragraph*{Logical Path to Work Accomplishment. Major Deliverables and Outcomes.} 
We anticipate the following outcomes - 
\begin{enumerate}[leftmargin=*,topsep=0pt,itemsep=-0.75ex,partopsep=1ex,parsep=1ex]
    \item
    \item
    \item
    \item Identify lessons learned and engage in future pathways and partnerships.
\end{enumerate}

\paragraph*{Estimated Cost of the Project.} 
We request a total of \$X,XXX,XXX for this project for a timeframe of N years.

%limiting tasks, deliverables to one page allows for more references to be included on the third page, for a 3 page limit

\begin{longtable}{|L{0.23\linewidth}|c|c|c|c|c|c|c|c|c|c|c|c|}
%       \captionsetup{margin=-0pt}
        \caption*{\textbf{Timeframe for Execution of Proposed Project. Schedule, Roles, and Responsibilities.}}
        \label{tab-timeline}\\
        \hline
        \multicolumn{1}{|c|}{\multirow{2}{*}{\textbf{TASKS}}}& 
        \multicolumn{4}{|c|}{\textbf{Y1}}&
        \multicolumn{4}{|c|}{\textbf{Y2}}&
        \multicolumn{4}{|c|}{\textbf{Y3}}
        \\
        \cline{2-13}
        &
        \textbf{Q1}& 
        \textbf{Q2}& 
        \textbf{Q3}& 
        \textbf{Q4}& 
        \textbf{Q1}& 
        \textbf{Q2}& 
        \textbf{Q3}& 
        \textbf{Q4}& 
        \textbf{Q1}& 
        \textbf{Q2}& 
        \textbf{Q3}& 
        \textbf{Q4}
        \\
        \hline
        I. 
        &\x\textbf{X}
        & 
        & 
        & 
        & 
        & 
        & 
        & 
        & 
        & 
        & 
        & 
        \\
        \hline
        II. 
        &\x\textbf{X} 
        & 
        & 
        & 
        & 
        & 
        & 
        & 
        & 
        & 
        & 
        & 
        \\
        \hline
        III. 
        &\x\textbf{X} 
        & 
        & 
        & 
        & 
        & 
        & 
        & 
        & 
        & 
        & 
        & 
        \\
        \hline
        IV. 
        &\x\textbf{X}
        & 
        & 
        & 
        & 
        & 
        & 
        & 
        & 
        & 
        & 
        & 
        \\
        \hline
        V. 
        &\x\textbf{X} 
        & 
        & 
        & 
        & 
        & 
        & 
        & 
        & 
        & 
        & 
        & 
        \\
        \hline
        VI. 
        &\x\textbf{X} 
        & 
        & 
        & 
        & 
        & 
        & 
        & 
        & 
        & 
        & 
        & 
        \\
        \hline
        VII. Crosscutting issues
        & 
        & 
        & 
        & 
        & 
        & 
        & 
        & 
        & 
        & 
        &\x\textbf{X} 
        &\x\textbf{X} 
        \\
        \hline
\end{longtable}

\newpage

%not typically needed in a preproposal, but good to do 
\begin{longtable}{|l|l|l|} %comment out for page limits
    \caption*{\textbf{Milestones.}}
    \label{tab-milestones}\\
    \hline
        \multicolumn{1}{|c|}{\textbf{Milestone}}
        &\multicolumn{1}{|c|}{\textbf{Deliverable}}
        &\multicolumn{1}{|c|}{\textbf{Completion}}\\
        \hline
        \begin{tabular}[c]{@{}l@{}}
            0. Project logistics
        \end{tabular}
        &
        \begin{tabular}[c]{@{}l@{}}
            0.1. Kick-off report, roles, and responsibilities.\\
            0.2. Annual reports.\\
            0.3. Agency administrative requirements.\\
            0.4. Final project report.\\
            0.5. Journal paper submission.\\
            0.6. Conference presentations.
        \end{tabular}
        &
        \begin{tabular}[c]{@{}l@{}}
            0.1. Y1-Q2\\
            0.2. Annually\\
            0.3. Quarterly\\
            0.4. Y3-Q4\\
            0.5. Annually\\
            0.6. Annually
        \end{tabular}
        \\
        \hline
        \begin{tabular}[c]{@{}l@{}}
            1.
        \end{tabular}
        &
        \begin{tabular}[c]{@{}l@{}}
            1.1. 

        \end{tabular}
        &
        \begin{tabular}[c]{@{}l@{}}
            1.1. YX-QN
        \end{tabular}
        \\
        \hline
        \begin{tabular}[c]{@{}l@{}}
            15. Crosscutting\\issues
        \end{tabular}
        &
        \begin{tabular}[c]{@{}l@{}}
            15.1. Identify lessons learned.\\
            15.2. Develop future pathways and partnerships.
        \end{tabular}
        &
        \begin{tabular}[c]{@{}l@{}}
            15.1. YEND-Q3\\
            15.2. YEND-Q4
        \end{tabular}
        \\
        \hline
\end{longtable}

\newpage

%remove page break if there are page limits
%\vspace{0.5\baselineskip}

\paragraph*{References}
\bibliographystyle{nsf}
\setlength{\bibhang}{0pt}
\bibliography{references}

%wrap figure around text
%move to wherever in the text 
%{l} fixes it on the left margin
%{L} floats
%\begin{wrapfigure}{l}{0.x\textwidth}
%       \includegraphics[width=0.(x-2)\textwidth]{}
%    \caption{}
%    \label{fig-}
%\end{wrapfigure}

\end{document}
