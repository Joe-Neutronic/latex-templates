%@TheDoctorRAB
%standard white paper/preproposal format
%
%%%%%
%
%REFERENCES
%
%neup.bst - numbered citations in order of appearance, short author list with et al in reference section
%nsf.bst - numbered citations in order of appearance, full author list in references section
%standard.bst - citations with author last name with et al for more than 2 authors; full author list in references section
%ans.bst is for ANS only. 
%
%author = {Lastname, Firstname and Lastname, Firstname and Lastname, Firstname} for all bst formats
%bst renders the author list itself
%
%author = {{Nuclear Regulatory Commission}} if the author is an organization, institution, etc., and not people
%
%title = {{}} for all
%
%for all - use \citep{-} - [1] or (Borrelli, 2021) in the text
%standard.bst \cite{-} - Borrelli (2021) in the text
%standard.bst lists references alphabetically
%the rest list numerically
%
%%%%%
\documentclass[11pt,letterpaper]{article}
\usepackage[lmargin=1in,rmargin=1in,tmargin=1in,bmargin=1in]{geometry}
\usepackage[pagewise]{lineno} %line numbering
\usepackage{setspace}
\usepackage{ulem} %strikethrough - do not \sout{\cite{}}
\usepackage[pdftex,dvipsnames]{xcolor,colortbl} %change font color
\usepackage{graphicx}
\usepackage{filecontents}
\usepackage{tablefootnote}
\usepackage{footnotehyper}
\usepackage{float}
%\usepackage{subfig}
\usepackage[yyyymmdd]{datetime} %date format
\renewcommand{\dateseparator}{.}
\graphicspath{{../img/}} %path to graphics
\setcounter{secnumdepth}{5} %set subsection to nth level

%fonts
\usepackage{times}
%arial - uncomment next two lines
%\usepackage{helvet}
%\renewcommand{\familydefault}{\sfdefault}

\usepackage{tabto} %general tabbed spacing
\usepackage{longtable} %need to put label at top under caption then \\ - use spacing
\usepackage[stable,hang,flushmargin]{footmisc} %footnotes in section titles and no indent; standard.bst
%\usepackage[round,semicolon]{natbib} %use 'numbers' for numbered citations; 'round' for () instead [] for inline citations
\usepackage[numbers,sort&compress]{natbib} %use 'numbers' for numbered citations; 'round' for () instead [] for inline citations; nsf.bst
\usepackage[inline]{enumitem} %use enumitem* to omit line breaks
\usepackage{boldline}
\usepackage{makecell}
\usepackage{booktabs}
\usepackage{amssymb}
\usepackage{gensymb}
\usepackage{amsmath}
\usepackage{physics}
\usepackage{tabularx}
\usepackage{multirow}
\usepackage{lscape}
\usepackage{array}
\usepackage{caption}
\usepackage{subcaption}
\usepackage[labelfont=bf]{caption}
\usepackage{chngcntr}
%\usepackage{hyperref}
\usepackage{sectsty}
\usepackage{textcomp}
\usepackage{lastpage}
\usepackage{xargs} %for \newcommandx
\usepackage[colorinlistoftodos,prependcaption,textsize=small]{todonotes} %makes colored boxes for commenting
\usepackage[toc,page]{appendix}
\usepackage[figure,table]{totalcount}
\usepackage[acronym,nomain,nonumberlist]{glossaries}
\makenoidxglossaries

\usepackage[singlelinecheck=false]{caption}
\captionsetup[table]{skip=7pt,labelformat={default},labelsep=period,name={Tab.}} %sets a space after table caption
\captionsetup[figure]{skip=7pt,labelformat={default},labelsep=period,name={Fig.}} %sets space above caption, 'figure' format

\usepackage{wrapfig}
\setlength{\intextsep}{0.20mm}
\setlength{\columnsep}{0.20mm}

%\usepackage{xr} %for revisions - will cross reference from one file to here
%\externaldocument{/path/to/auxfilename} %aux file needed

\newcommand{\edit}[1]{\textcolor{blue}{#1}} %shortcut for changing font color on revised text
\newcommand{\fn}[1]{\footnote{#1}} %shortcut for footnote tag
\newcommand*\sq{\mathbin{\vcenter{\hbox{\rule{.3ex}{.3ex}}}}} %makes a small square as a separator $\sq$
\newcommand{\sk}[1]{\sout{#1}} %shortcut for strikethrough
\newcommand{\x}{\cellcolor{lightgray}} %use to shade in table cell
\newcommand{\acf}{\acrfull} %full acronym
\newcommand{\acl}{\acrlong} %long acronym
\newcommand{\acs}{\acrshort} %short acronym

\newcommandx{\que}[2][1=]{\todo[linecolor=red,backgroundcolor=red!25,bordercolor=red,#1]{#2}} %query
\newcommandx{\sug}[2][1=]{\todo[linecolor=blue,backgroundcolor=blue!25,bordercolor=blue,#1]{#2}} %suggested change
\newcommandx{\cmt}[2][1=]{\todo[linecolor=OliveGreen,backgroundcolor=OliveGreen!25,bordercolor=OliveGreen,#1]{#2}} %comment
\newcommandx{\omt}[2][1=]{\todo[linecolor=Plum,backgroundcolor=Plum!25,bordercolor=Plum,#1]{#2}} %omit

\newcolumntype{L}[1]{>{\raggedright\let\newline\\\arraybackslash\hspace{0pt}}p{#1}} %uses \raggedright with p{} in table column

\makeatletter
\renewcommand\tableofcontents{%
    \@starttoc{toc}%
}
\makeatother

\makeatletter
\renewcommand\listoffigures{%
    \@starttoc{lof}%
}
\makeatother

\makeatletter
\renewcommand\listoftables{%
    \@starttoc{lot}%
}
\makeatother

\makeatletter
\newcommand*\ftp{\fontsize{16.5}{17.5}\selectfont}
\makeatother

\makeatletter
\renewcommand\section{%
    \@startsection{section}{1}{\z@ }{0.50\baselineskip}{0.25\baselineskip}
    {\normalfont \normalsize \bfseries}}%

\makeatletter
\renewcommand\paragraph{%
    \@startsection{paragraph}{4}{\z@ }{0.20\baselineskip}{-0.5em}
    {\normalfont \normalsize \bfseries}}%

\makeatletter
\renewcommand\subparagraph{%
    \@startsection{subparagraph}{5}{\z@ }{0.10\baselineskip}{-0.5em}
    {\normalfont \normalsize \itshape }}%

\makeatletter
\renewcommand\subsection{%
    \@startsection{subsection}{2}{\z@ }{0.45\baselineskip}{0.25\baselineskip}
    { \large \normalsize \bfseries}}%
    
\setlength{\bibsep}{0pt} %sets space between references
\renewcommand{\bibsection}{} %suppresses large 'references' heading
\renewcommand\bibpreamble{\vspace{-0.30\baselineskip}} %sets spacing after heading if not using default references heading

\usepackage{fancyhdr}
\pagestyle{fancy}
\fancyhf{} %move page number to bottom right
\renewcommand{\headrulewidth}{0pt} %set line thickness in header; uncomment as is to remove line
%\lhead{\scriptsize Agency}
%\chead{\scriptsize \textit{Title}}
%\rhead{\scriptsize \today}
\rfoot{\thepage}

\begin{filecontents}{references.bib} 
    @misc{
        nye15a,
        author = {Nye, Jr., Joseph S.},
        title = {{The World Needs an Arms-control Treaty for Cybersecurity}},
        year = {2015},
        note = {{The Washington Post}}
    }
    @article{
        mil20a,
        author = {Miller, James N.},
        title = {No to no first use — for now},
        journal = {Bulletin of the Atomic Scientists},
        volume = {76},
        pages = {8},
        year  = {2020}
    }
    @incollection{
        mil17a,
        author = {Miller, Steven E.},
        title = {{Cyber Threats, Nuclear Analogies? Divergent Trajectories in Adapting to New Dual-Use Technologies}},
        booktitle = {Understanding Cyber Conflict: 14 Analogies},
        publisher = {Georgetown University Press},
        year = {2017},
        editor = {Perkovich, George and Levite, Ariel E.},
        pages = {161}
    }
    @conference{
        ben20a,
        author = {Benjamin, Jacob and Haney, Michael},
        title = {{Nonproliferation of Cyber Weapons}},
        year = { 2020},
        organization = {Symposium on Cyber Warfare, Cyber Defense, \& Cyber Security },
        address = {Las Vegas, Nevada}
    }
\end{filecontents}

%\newacronym{nrc}{NRC}{United States Nuclear Regulatory Commission}
%\newacronym{}{}{}

%spacing options
%\onehalfspacing %linespacing
%\setstretch{1.05} %linespacing
%\spacing{1.25} %equivalent to 1.5 line spacing in Word

%linenumbering
%\linenumbers %toggle line numbers
%\pagewiselinenumbers %reset line numbers on new page
%\modulolinenumbers[3] %line numbering interval

\begin{document}

{\centering 
    \textbf{Title\\
    Call/Technical Scope\\
    }
    Name (PI) - Affiliation\\
    Name (co-PI) - Affiliation
\par
}

\subsection*{Summary of the Proposed Project} 

%\begin{wrapfigure}{l}{0.42\textwidth}
%    \includegraphics[width=0.40\textwidth]{flowsheet.jpg}
%    \captionsetup{justification=centering,belowskip=-0.7mm}
%    \caption{Pyroprocessing flowsheet.}
%    \label{fig-flowsheet}
%\end{wrapfigure}

\subsection*{Motivation} 
\paragraph*{Context.} 

\paragraph*{Something else.} 

\subsection*{Overall Objective} 

\subsection*{Research Focus} 
\noindent Project workscopes are summarized. 
\begin{enumerate}[leftmargin=*,topsep=0pt,itemsep=-1ex,partopsep=1ex,parsep=1ex]
    \item\textit{Title (Lead - )}
        \begin{itemize}[topsep=-1ex,itemsep=-1ex,partopsep=1ex,parsep=1ex]
            \item 
            \item 
        \end{itemize}
    \item\textit{}
        \begin{itemize}[topsep=-1ex,itemsep=-1ex,partopsep=1ex,parsep=1ex]
            \item 
            \item 
        \end{itemize}
    \item\textit{}
        \begin{itemize}[topsep=-1ex,itemsep=-1ex,partopsep=1ex,parsep=1ex]
            \item 
            \item  
        \end{itemize}
    \item\textit{Conclusions - Crosscutting issues (Full team)}
        \begin{itemize}[topsep=-1ex,itemsep=-1ex,partopsep=1ex,parsep=1ex]
            \item Identify lessons learned.
            \item Develop future research pathways.
        \end{itemize}
\end{enumerate}

\subsection*{Importance and Relevance to Objectives}  
\paragraph*{Background.} 
\subparagraph*{} 

\subparagraph*{} 

\subparagraph*{} 

\paragraph*{} 

\paragraph*{} 

\subparagraph*{} 

\subparagraph*{}

\paragraph*{Unique project features.} 

\subsection*{Logical Path, Workscope, Description of Tasks} 
\paragraph*{Workscope 1 - Title.} 
\subparagraph*{Task I. Title.} 

\subparagraph*{Task II. Title.} 

\subparagraph*{Task III. Title.} 

\subparagraph*{Task IV. Title.} 

\paragraph*{Workscope 2 - Title.} 
\subparagraph*{Task V. Title.} 

\paragraph*{Workscope 6. Conclusions.}
\subparagraph*{Task N. Crosscutting Issues.} 
\noindent Lessons learned will be identified for project outcomes. Selected additional discussion will cover, but not limited to, the following -
\begin{enumerate}[leftmargin=*,topsep=0pt,itemsep=-1ex,partopsep=1ex,parsep=1ex]
    \item 
    \item  
    \item 
\end{enumerate}

\paragraph*{Logical Path to Work Accomplishment. Major Deliverables and Outcomes.} 
\noindent The following outcomes are anticipated -
\begin{enumerate}[leftmargin=*,topsep=0pt,itemsep=-1ex,partopsep=1ex,parsep=1ex]
    \item
    \item
    \item 
    \item
    \item
\end{enumerate}

%\noindent\textbf{Estimated Cost of the Project.} The maximum amount of \$400,000 is requested for the 2 year time period at \$200,000 per year.
%

\begin{longtable}{|L{0.27\linewidth}|c|c|c|c|c|c|c|c|c|c|c|c|}
    \caption*{\textbf{Timeframe for Execution of Proposed Project. Schedule, Roles, and Responsibilities.}}
    \label{tab-timeframe}\\
        \hline
        \multicolumn{1}{|c|}{\multirow{2}{*}{\textbf{TASKS}}}& 
        \multicolumn{4}{|c|}{\textbf{Y1}}&
        \multicolumn{4}{|c|}{\textbf{Y2}}&
        \multicolumn{4}{|c|}{\textbf{Y3}}
        \\
        \cline{2-13}
        &
        \textbf{Q1}& 
        \textbf{Q2}& 
        \textbf{Q3}& 
        \textbf{Q4}& 
        \textbf{Q1}& 
        \textbf{Q2}& 
        \textbf{Q3}& 
        \textbf{Q4}& 
        \textbf{Q1}& 
        \textbf{Q2}& 
        \textbf{Q3}& 
        \textbf{Q4}
        \\
        \hline
        I. 
        &\x\textbf{I} 
        &
        &
        &
        &
        &
        &
        &
        &
        &
        &
        &
        \\
        \hline
        II. 
        &\x\textbf{II} 
        &\x\textbf{II} 
        &
        &
        &
        &
        &
        &
        &
        & 
        & 
        & 
        \\
        \hline
        III. 
        &\x\textbf{III} 
        &
        &
        &
        &
        &
        &
        &
        &
        &
        & 
        & 
        \\
        \hline
        IV. 
        & 
        &\x\textbf{IV} 
        & 
        &
        &
        &
        &
        &
        &
        &
        &
        & 
        \\
        \hline
        V. 
        & 
        &\x\textbf{V} 
        &\x\textbf{V} 
        &\x\textbf{V}
        &
        &
        &
        &
        &
        &
        &
        &
        \\
        \hline
        VI. 
        & 
        & 
        & 
        &\x\textbf{VI}
        & 
        &
        &
        &
        &
        & 
        &
        &
        \\
        \hline
        VII. 
        & 
        & 
        & 
        &\x\textbf{VII}
        & 
        &
        &
        &
        &
        & 
        &
        &
        \\
        \hline
        VIII. 
        & 
        & 
        & 
        & 
        &\x\textbf{VIII} 
        & 
        &
        &
        &
        &
        &
        &
        \\
        \hline
        IX. 
        & 
        & 
        & 
        &\x\textbf{IX} 
        &\x\textbf{IX} 
        & 
        &
        &
        &
        &
        &
        &
        \\
        \hline
        X. 
        & 
        & 
        &
        &
        &
        &
        & 
        & 
        &\x\textbf{X} 
        &\x\textbf{X} 
        &\x\textbf{X}
        &\x\textbf{X}
        \\
        \hline
        XI. Crosscutting issues 
        & 
        & 
        &
        &
        &
        &
        & 
        & 
        & 
        &
        &\x\textbf{VII}
        &\x\textbf{VII} 
        \\
        \hline
\end{longtable}

\vspace{-\baselineskip}

\begin{longtable}{|l|l|l|}
    \caption*{\textbf{Milestones.}}
    \label{tab-milestones}\\
        \hline
        \multicolumn{1}{|c|}{\textbf{Milestone}}
        &\multicolumn{1}{|c|}{\textbf{Deliverable}}
        &\multicolumn{1}{|c|}{\textbf{Completion}}\\
        \hline
        \begin{tabular}[c]{@{}l@{}}
        0. Project\\logistics
        \end{tabular}
        &
        \begin{tabular}[c]{@{}l@{}}
        0.1. Kick-off report, roles, and responsibilities.\\
        0.2. Annual reports.\\
        0.3. Agency administrative requirements.\\
        0.4. Final project report.\\
        0.5. Journal paper submission.\\
        0.6. Professional society conference presentations.
        \end{tabular}
        & 
        \begin{tabular}[c]{@{}l@{}}
        0.1. Y1-Q2\\
        0.2. Annually\\
        0.3. Quarterly\\
        0.4. Y2-Q4\\
        0.5. Annually\\
        0.6. Annually
        \end{tabular}
        \\
        \hline
        \begin{tabular}[c]{@{}l@{}}
        1. 
        \end{tabular}
        &
        \begin{tabular}[c]{@{}l@{}}
        1.1.  \\
        1.2. 
        \end{tabular}
        &
        \begin{tabular}[c]{@{}l@{}}
        1.1.  \\
        1.2.  
        \end{tabular}
        \\
        \hline
        \begin{tabular}[c]{@{}l@{}}
        2. 
        \end{tabular}
        &
        \begin{tabular}[c]{@{}l@{}}
        2.1. \\
        2.2. \\
        2.3. 
        \end{tabular}
        &
        \begin{tabular}[c]{@{}l@{}}
        2.1. \\
        2.2. \\
        2.3. 
        \end{tabular}
        \\
        \hline
        \begin{tabular}[c]{@{}l@{}}
        3. 
        \end{tabular}
        &
        \begin{tabular}[c]{@{}l@{}}
        3.1. 
        \end{tabular}
        &
        \begin{tabular}[c]{@{}l@{}}
        3.1. 
        \end{tabular}
        \\
        \hline
        \begin{tabular}[c]{@{}l@{}}
        4. 
        \end{tabular}
        &
        \begin{tabular}[c]{@{}l@{}}
        4.1. \\
        4.2. \\
        4.3. 
        \end{tabular}
        &
        \begin{tabular}[c]{@{}l@{}}
        4.1. \\
        4.2. 
        \end{tabular}
        \\
        \hline
        \begin{tabular}[c]{@{}l@{}}
        5. 
        \end{tabular}
        &
        \begin{tabular}[c]{@{}l@{}}
        5.1. \\
        5.2. \\
        5.3. \\
        5.4. \\
        5.5. 
        \end{tabular}
        & 
        \begin{tabular}[c]{@{}l@{}}
        5.1. \\
        5.2. \\
        5.3. \\
        5.4. \\
        5.5. 
        \end{tabular}
        \\
        \hline
        \begin{tabular}[c]{@{}l@{}}
        6. 
        \end{tabular}
        &
        \begin{tabular}[c]{@{}l@{}}
        6.1. \\
        6.2.  
        \end{tabular}
        &
        \begin{tabular}[c]{@{}l@{}}
        6.1. \\
        6.2. 
        \end{tabular}
        \\
        \hline
        \begin{tabular}[c]{@{}l@{}}
        7. 
        \end{tabular}
        &
        \begin{tabular}[c]{@{}l@{}}
        7.1. 
        \end{tabular}
        &
        \begin{tabular}[c]{@{}l@{}}
        7.1. 
        \end{tabular}
        \\
        \hline
        \begin{tabular}[c]{@{}l@{}}
        8. 
        \end{tabular}
        &
        \begin{tabular}[c]{@{}l@{}}
        8.1. \\
        8.2. \\
        8.3. 
        \end{tabular}
        &
        \begin{tabular}[c]{@{}l@{}}
        8.1. \\
        8.2. \\
        8.3. 
        \end{tabular}
        \\
        \hline
        \begin{tabular}[c]{@{}l@{}}
        9. 
        \end{tabular}
        &
        \begin{tabular}[c]{@{}l@{}}
        9.1. \\
        9.2. \\
        9.3. 
        \end{tabular}
        &
        \begin{tabular}[c]{@{}l@{}}
        9.1. \\
        9.2. \\
        9.3. 
        \end{tabular}
        \\
        \hline
        \begin{tabular}[c]{@{}l@{}}
        10. 
        \end{tabular}
        &
        \begin{tabular}[c]{@{}l@{}}
        10.1. \\
        10.2. \\
        10.3. \\
        10.4. \\
        10.5. \\
        10.6. 
        \end{tabular}
        &
        \begin{tabular}[c]{@{}l@{}}
        10.1. \\
        10.2. \\
        10.3. \\
        10.4. \\
        10.5. \\
        10.6. 
        \end{tabular}
        \\
        \hline
        \begin{tabular}[c]{@{}l@{}}
        11. Discuss\\crosscutting\\issues
        \end{tabular}
        &
        \begin{tabular}[c]{@{}l@{}}
        11.1. Identify lessons learned.\\
        11.2. Develop future research and educational pathways.
        \end{tabular}
        &
        \begin{tabular}[c]{@{}l@{}}
        11.1. Y3-Q3\\
        11.2. Y3-Q4
        \end{tabular}
        \\
        \hline
\end{longtable}

\paragraph*{Contribution to advancing state of the art in a, b, and c.}
\noindent This research plan and deliverables will advance the state-of-the-art in a, b, and, c as it is expected to - (1), (2), (3), and (4). 

\paragraph*{Description of facilities to be utilized to execute scope.} 

\paragraph*{Challenges to accomplish tasks and planned mitigations.} 
\noindent The PI is based at a United States Research University. The co-PIs are based at \textit{Affiliations}. This proposed project supports the goals and research interests of the institutions in the field of \textit{major fields}.  

\vspace{-\baselineskip}

\begin{longtable}{|L{0.45\linewidth}|L{0.45\linewidth}|}
    \caption*{}
    \label{tab-risk}\\
        \hline
        \multicolumn{1}{|c|}{\textbf{Challenges to accomplish tasks}}
        &\multicolumn{1}{|c|}{\textbf{Planned mitigations}}\\
        \hline
        
        & 
        \\
        \hline
        
        &  
        \\
        \hline
         
        &  
        \\
        \hline
        
        & 
        \\
        \hline
\end{longtable}

\paragraph*{Information, data, and plans.} 
\noindent All information has been presented in the technical scope.

\paragraph*{Quality assurance.} 
\noindent Our project team have practiced a QA approach, consistent with the graded approach established by the DOE Technical Integration Office, per the NEUP website and any additional requirements deemed necessary during contracting. All projects undergo safety review by experts. We have followed this QA expectation in all ongoing awards.

\paragraph*{Mandatory requirements.} 
\noindent We abide by current and pending information on nuclear-related federal funding contained in Form 3204. The Mandatory, Go/No-go, requirements will be coordinated with the institutional Office of Sponsored Research, the general counsel, and the Office of Technology Transfer. Each PI noted below will assume responsibility as the point of contact.

\vspace{-\baselineskip}

\begin{longtable}{|c|c|c|c|}
    \caption*{}
    \label{tab-mandatory-requirements}\\
        \hline
        \textbf{No.}
        &\textbf{Requirement}
        &\textbf{Responsibility}
        &\textbf{Evaluation}\\
        \hline
        1
        &Commitment to reporting and budget requirements
        &All PIs
        &Go/No-go 
        \\
        \hline
        2
        &10 CFR 851 Worker Safety and Health Program
        &All PIs
        &Go/No-go 
        \\
        \hline
        3
        &Export Control
        &All PIs
        &Go/No-go 
        \\
        \hline
        4
        &Standard Research Subcontract
        &All PIs
        &Go/No-go 
        \\
        \hline
        5
        &Quality Assurance
        &All PIs
        &Go/No-go 
        \\
        \hline
        6
        &Commitment to prepare additional contract elements
        &All PIs
        &Go/No-go 
        \\
        \hline
\end{longtable}

\vspace{-0.50\baselineskip}

\noindent\textbf{References}
\bibliographystyle{neup}
\setlength{\bibhang}{0pt}
\bibliography{references}

\end{document}
