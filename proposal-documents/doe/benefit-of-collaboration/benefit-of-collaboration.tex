%@TheDoctorRAB
%standard white paper/preproposal format
%
%%%%%
%
%REFERENCES
%
%neup.bst - numbered citations in order of appearance, short author list with et al in reference section
%nsf.bst - numbered citations in order of appearance, full author list in references section
%standard.bst - citations with author last name with et al for more than 2 authors; full author list in references section
%ans.bst is for ANS only. 
%
%author = {Lastname, Firstname and Lastname, Firstname and Lastname, Firstname} for all bst formats
%bst renders the author list itself
%
%author = {{Nuclear Regulatory Commission}} if the author is an organization, institution, etc., and not people
%
%title = {{}} for all
%
%for all - use \citep{-} - [1] or (Borrelli, 2021) in the text
%standard.bst \cite{-} - Borrelli (2021) in the text
%standard.bst lists references alphabetically
%the rest list numerically
%
%%%%%
\documentclass[11pt,letterpaper]{article}
\usepackage[lmargin=1in,rmargin=1in,tmargin=1in,bmargin=1in]{geometry}
\usepackage[pagewise]{lineno} %line numbering
\usepackage{setspace}
\usepackage{ulem} %strikethrough - do not \sout{\cite{}}
\usepackage[pdftex,dvipsnames]{xcolor,colortbl} %change font color
\usepackage{graphicx}
\usepackage{filecontents}
\usepackage{tablefootnote}
\usepackage{footnotehyper}
%\usepackage{subfig}
\usepackage[yyyymmdd]{datetime} %date format
\renewcommand{\dateseparator}{.}
\graphicspath{{../img/}} %path to graphics
\setcounter{secnumdepth}{5} %set subsection to nth level

%fonts
\usepackage{times}
%arial - uncomment next two lines
%\usepackage{helvet}
%\renewcommand{\familydefault}{\sfdefault}

\usepackage{tabto} %general tabbed spacing
\usepackage{longtable} %need to put label at top under caption then \\ - use spacing
\usepackage[stable,hang,flushmargin]{footmisc} %footnotes in section titles and no indent; standard.bst
%\usepackage[round,semicolon]{natbib} %use 'numbers' for numbered citations; 'round' for () instead [] for inline citations
\usepackage[numbers,sort&compress]{natbib} %use 'numbers' for numbered citations; 'round' for () instead [] for inline citations; nsf.bst
\usepackage{enumitem}
\usepackage{boldline}
\usepackage{makecell}
\usepackage{booktabs}
\usepackage{amssymb}
\usepackage{gensymb}
\usepackage{amsmath}
\usepackage{physics}
\usepackage{tabularx}
\usepackage{multirow}
\usepackage{lscape}
\usepackage{array}
\usepackage{caption}
\usepackage{subcaption}
\usepackage[labelfont=bf]{caption}
\usepackage{chngcntr}
%\usepackage{hyperref}
\usepackage{sectsty}
\usepackage{textcomp}
\usepackage{lastpage}
\usepackage{xargs} %for \newcommandx
\usepackage[colorinlistoftodos,prependcaption,textsize=small]{todonotes} %makes colored boxes for commenting
\usepackage[toc,page]{appendix}
\usepackage[figure,table]{totalcount}
\usepackage[acronym,nomain,nonumberlist]{glossaries}
\makenoidxglossaries

\usepackage[singlelinecheck=false]{caption}
\captionsetup[table]{skip=7pt,labelformat={default},labelsep=period,name={Tab.}} %sets a space after table caption
\captionsetup[figure]{skip=7pt,labelformat={default},labelsep=period,name={Fig.}} %sets space above caption, 'figure' format

\usepackage{wrapfig}
\setlength{\intextsep}{0.20mm}
\setlength{\columnsep}{0.20mm}

%\usepackage{xr} %for revisions - will cross reference from one file to here
%\externaldocument{/path/to/auxfilename} %aux file needed

\newcommand{\edit}[1]{\textcolor{blue}{#1}} %shortcut for changing font color on revised text
\newcommand{\fn}[1]{\footnote{#1}} %shortcut for footnote tag
\newcommand*\sq{\mathbin{\vcenter{\hbox{\rule{.3ex}{.3ex}}}}} %makes a small square as a separator $\sq$
\newcommand{\sk}[1]{\sout{#1}} %shortcut for strikethrough
\newcommand{\x}{\cellcolor{lightgray}} %use to shade in table cell
\newcommand{\acf}{\acrfull} %full acronym
\newcommand{\acl}{\acrlong} %long acronym
\newcommand{\acs}{\acrshort} %short acronym

\newcommandx{\que}[2][1=]{\todo[linecolor=red,backgroundcolor=red!25,bordercolor=red,#1]{#2}} %query
\newcommandx{\sug}[2][1=]{\todo[linecolor=blue,backgroundcolor=blue!25,bordercolor=blue,#1]{#2}} %suggested change
\newcommandx{\cmt}[2][1=]{\todo[linecolor=OliveGreen,backgroundcolor=OliveGreen!25,bordercolor=OliveGreen,#1]{#2}} %comment
\newcommandx{\omt}[2][1=]{\todo[linecolor=Plum,backgroundcolor=Plum!25,bordercolor=Plum,#1]{#2}} %omit

\newcolumntype{L}[1]{>{\raggedright\let\newline\\\arraybackslash\hspace{0pt}}p{#1}} %uses \raggedright with p{} in table column

\makeatletter
\renewcommand\tableofcontents{%
    \@starttoc{toc}%
}
\makeatother

\makeatletter
\renewcommand\listoffigures{%
    \@starttoc{lof}%
}
\makeatother

\makeatletter
\renewcommand\listoftables{%
    \@starttoc{lot}%
}
\makeatother

\makeatletter
\newcommand*\ftp{\fontsize{16.5}{17.5}\selectfont}
\makeatother

\makeatletter
\renewcommand\section{%
    \@startsection{section}{1}{\z@ }{0.50\baselineskip}{0.25\baselineskip}
    {\normalfont \normalsize \bfseries}}%

\makeatletter
\renewcommand\paragraph{%
    \@startsection{paragraph}{4}{\z@ }{0.25\baselineskip}{-1em}
    {\normalfont \normalsize \bfseries}}%

\makeatletter
\renewcommand\subparagraph{%
    \@startsection{subparagraph}{5}{\z@ }{0.20\baselineskip}{-1em}
    {\normalfont \normalsize \itshape }}%

\makeatletter
\renewcommand\subsection{%
    \@startsection{subsection}{2}{\z@ }{0.45\baselineskip}{0.25\baselineskip}
    { \large \normalsize \bfseries}}%
    
\setlength{\bibsep}{0pt} %sets space between references
\renewcommand{\bibsection}{} %suppresses large 'references' heading
\renewcommand\bibpreamble{\vspace{-0.30\baselineskip}} %sets spacing after heading if not using default references heading

\usepackage{fancyhdr}
\pagestyle{fancy}
\fancyhf{} %move page number to bottom right
\renewcommand{\headrulewidth}{0pt} %set line thickness in header; uncomment as is to remove line
%\lhead{\scriptsize Agency}
%\chead{\scriptsize \textit{Title}}
%\rhead{\scriptsize \today}
\rfoot{\thepage}

\begin{filecontents}{references.bib} 
    @misc{
        nye15a,
        author = {Nye, Jr., Joseph S.},
        title = {{The World Needs an Arms-control Treaty for Cybersecurity}},
        year = {2015},
        note = {{The Washington Post}}
    }
    @article{
        mil20a,
        author = {Miller, James N.},
        title = {No to no first use — for now},
        journal = {Bulletin of the Atomic Scientists},
        volume = {76},
        pages = {8},
        year  = {2020}
    }
    @incollection{
        mil17a,
        author = {Miller, Steven E.},
        title = {{Cyber Threats, Nuclear Analogies? Divergent Trajectories in Adapting to New Dual-Use Technologies}},
        booktitle = {Understanding Cyber Conflict: 14 Analogies},
        publisher = {Georgetown University Press},
        year = {2017},
        editor = {Perkovich, George and Levite, Ariel E.},
        pages = {161}
    }
    @conference{
        ben20a,
        author = {Benjamin, Jacob and Haney, Michael},
        title = {{Nonproliferation of Cyber Weapons}},
        year = { 2020},
        organization = {Symposium on Cyber Warfare, Cyber Defense, \& Cyber Security },
        address = {Las Vegas, Nevada}
    }
\end{filecontents}

%\newacronym{nrc}{NRC}{United States Nuclear Regulatory Commission}
%\newacronym{}{}{}

%spacing options
%\onehalfspacing %linespacing
%\setstretch{1.05} %linespacing
%\spacing{1.25} %equivalent to 1.5 line spacing in Word

%linenumbering
%\linenumbers %toggle line numbers
%\pagewiselinenumbers %reset line numbers on new page
%\modulolinenumbers[3] %line numbering interval

\begin{document}

{\centering 
    \textbf{BENEFIT OF COLLABORATORS\\
    Title \\
    Call\\
    }
    Personnel - Affiliation 
\par
}

\subsection*{Summary of the Proposed Project} 

\subsection*{Motivation} 
\paragraph*{Context.} 

\paragraph*{Challenges to nuclear materials accounting.} 

\subsection*{Overall Objective} 

\subsection*{Research Focus} 
\noindent Project workscopes are summarized. 
\begin{enumerate}[leftmargin=*,topsep=0pt,itemsep=-1ex,partopsep=1ex,parsep=1ex]
    \item\textit{Title (Lead - )}
        \begin{itemize}[topsep=-1ex,itemsep=-1ex,partopsep=1ex,parsep=1ex]
            \item 
            \item 
        \end{itemize}
    \item\textit{}
        \begin{itemize}[topsep=-1ex,itemsep=-1ex,partopsep=1ex,parsep=1ex]
            \item 
            \item 
        \end{itemize}
    \item\textit{}
        \begin{itemize}[topsep=-1ex,itemsep=-1ex,partopsep=1ex,parsep=1ex]
            \item 
            \item  
        \end{itemize}
    \item\textit{Conclusions - Crosscutting issues (Full team)}
        \begin{itemize}[topsep=-1ex,itemsep=-1ex,partopsep=1ex,parsep=1ex]
            \item Identify lessons learned.
            \item Develop future research pathways.
        \end{itemize}
\end{enumerate}

\subsection*{Team} This research will facilitate collaboration among current industry experts with extensive \textit{fill in phrase about team expertise.} The project team consists of the following members. The contribution of each is noted. The corresponding facilities are briefly described thereafter.

\paragraph*{Prof./Dr. Firstname Lastname (Ph.D., University, Department).} 
\noindent Prof/Dr. Lastname is a \textit{title} at \textit{Current affiliation}, etc. His/Her/Their research focus is in \textit{area of expertise}. Prof./Dr. Name is the PI of this project and will be responsible for progress and achievement of all objectives, tasks, and deliverables. He/She/They has expertise in \textit{topics contained in text of narrative} from \textit{where/how this expertise was acquired}. He/She/They teach \textit{relevant courses} at \textit{Current affiliation}.

\paragraph*{Prof./Dr. Firstname Lastname (Ph.D., University, Department).} 
\noindent Prof./Dr. Lastname is a \textit{title} at \textit{Affiliation}, etc. His/Her/Their research focus is in \textit{area of expertise}. Prof./Dr. Name is the co-PI of this project and will contibute expertise in \textit{related project tasks}. He/She/They also is/was \textit{related experience} from \textit{where/how this expertise was acquired}. He/She/They teach \textit{relevant courses} at \textit{Affiliation}.

\subsection*{Important Research Facility/Equipment}
\noindent The Important Research Facility/Equipment is \textit{description as relevant to project.}

\subsection*{Work Project Documentation, Quality Control} 
\noindent The PI has a practiced a QA approach, consistent with that established by the DOE Technical Integration Office, per NEUP website and any additional requirements deemed necessary during contracting. All projects undergo safety review by experts. We have followed this QA expectation in all ongoing awards. We abide by current and pending information on nuclear-related federal funding contained in Form 3204. The Mandatory, Go/No-go, requirements will be coordinated with The University of Idaho Office of Sponsored Research, the general counsel and Technology Transfer. The PI assumes responsibility as the POC and follow these requirements: 1) Commitment to reporting and budget requirements, 2) 10CFR851 Worker Safety and Health Program, 3) Export Control and 4) Quality Assurance.

\end{document}
