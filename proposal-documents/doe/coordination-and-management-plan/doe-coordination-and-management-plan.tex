\documentclass[11pt,letterpaper]{article}
\usepackage[lmargin=1in,rmargin=1in,tmargin=1in,bmargin=1in]{geometry}
\usepackage[pagewise]{lineno} %line numbering
\usepackage{setspace}
\usepackage{ulem} %strikethrough
\usepackage{xcolor,colortbl} %change font color
\usepackage{graphicx}
\usepackage{filecontents}
\usepackage{tablefootnote}
\usepackage{subfig}
\usepackage[yyyymmdd]{datetime}
\renewcommand{\dateseparator}{.}
\graphicspath{{../img/}}
\setcounter{secnumdepth}{5} %set subsection to nth level
\usepackage{times}
\usepackage{enumitem}
\usepackage{float}
\usepackage{multirow}
\usepackage{tabularx}
\usepackage{tabulary}
\usepackage{gensymb}
\usepackage{longtable}
%\usepackage[labelfont=bf]{caption}

\usepackage[singlelinecheck=false]{caption}
\captionsetup[table]{skip=0pt} %sets a space after table caption
\captionsetup[figure]{skip=0pt,labelformat={default},labelsep=period,name={Fig.}} %sets space above caption, 'figure' format

\usepackage{wrapfig}
\setlength{\intextsep}{0.20mm}
\setlength{\columnsep}{0.20mm}

\usepackage[numbers,sort&compress]{natbib} 
%'numbers' - numbered citations
%'round' - (1) instead [1]
%'sort&compress - [1-3] instead [1,2,3]
\setlength{\bibsep}{0pt} %sets space between references
\renewcommand{\bibsection}{} %suppresses large 'references' heading
\renewcommand\bibpreamble{\vspace{-0.2\baselineskip}} %sets spacing after heading if not using default references heading

\newcommand{\edit}[1]{\textcolor{blue}{#1}} %shortcut for changing font color on revised text
\newcommand{\sk}[1]{\sout{#1}} %shortcut for strikethrough
\newcommand{\fn}[1]{\footnote{#1}} %shortcut for footnote tag
\newcommand*\sq{\mathbin{\vcenter{\hbox{\rule{.3ex}{.3ex}}}}} %makes a small square as a separator $\sq$
\newcommand{\x}{\cellcolor{lightgray}} %use to shade in table cell

\newcolumntype{L}[1]{>{\raggedright\let\newline\\\arraybackslash\hspace{0pt}}p{#1}} %uses \raggedright with p{} in table column

\makeatletter
\renewcommand\paragraph{%
    \@startsection{paragraph}{4}{\z@ }{0.35\baselineskip}{-1em}
    {\normalfont \normalsize \bfseries}}%

\makeatletter
\renewcommand\subparagraph{%
    \@startsection{subparagraph}{5}{\z@ }{0.20\baselineskip}{-1em}
    {\normalfont \normalsize \itshape }}%

\makeatletter
\renewcommand\subsection{%
    \@startsection{subsection}{2}{\z@ }{0.50\baselineskip}{0.10\baselineskip}
    { \normalfont \bfseries}}%

\usepackage{fancyhdr}
\pagestyle{fancy}
\fancyhf{} %move page number to bottom right
\renewcommand{\headrulewidth}{0pt} %set line thickness in header
%\lhead{\scriptsize Name}
%\chead{\scriptsize Title, subject, agency, etc.}
%\rhead{\scriptsize \today}
\rfoot{\thepage}

\begin{document}

{\centering 
    \textbf{COORDINATION AND MANAGEMENT PLAN\\
    Title\\
    Call/Technical Scope\\
    }
    Name (PI) - Affiliation\\
    Name (co-PI) - Affiliation
\par
}

\subsection*{Project Collaborators}
\noindent The proposed project is a collaboration between Affiliation I and Affiliation II. Affiliation I is the lead institution and will coordinate the overall project. Further information regarding team member expertise and contributions are contained in the Benefits of Collaboration and Capabilities document.

\subsection*{PI Roles and Responsibilities}
\noindent Affiliation I is the lead institution for the project. This project team consists of the Principal Investigator (PI), co-Principal Investigator (co-PI), and list others. The PI will oversee administrative duties and manage the project throughout the project duration. The PI and team all hold responsibility for implementing the proposed milestones and deliverables as defined by the schedule in the Technical Narrative. The research proposed in this project will be conducted by graduate students under the guidance, direction, and mentorship of the PI and co-PIs. Periodic meetings will be held to maintain satisfactory progress on this project. The PI and team will also serve on thesis committees for the graduate students.

\subsection*{Communication Plans}
\noindent A kickoff conference will be held within the first quarter of the project period. A central, online repository will be created and maintained by Affiliation I. This will be the hub for communication and exchange of relevant documents. Regular project meetings will be held monthly for team members for performance updates and related project issues to address as they arise. A full project review will be conducted annually and include a written summary of progress. Reporting to the United States Department of Energy will include quarterly reports, annual report, and technical deliverables defined in the Technical Narrative.

\subsection*{Process and Procedures for Making Decisions and Resolving Conflicts}
\noindent Decision-making is based on the contract signed by Affiliation I and the United States Department of Energy. While major changes in the research direction are rare, this will first be coordinated with the sponsor if necessary, with a subsequent meeting for the team members to hold discussions and arrive at a consensus. If necessary, a majority vote amongst the PI and team will be held. PI and team all hold an equal vote, regardless of the funding distribution. Based on this, the PI will contact the Federal and Technical Point-of-Contact to inform them of the decision. The PI will inform the Federal and Technical Point-of-Contact regarding research progress, new challenges, and relevant solutions.

\subsection*{Publications}
\noindent Quarterly and annual reporting requirements will be prepared by the PI with supporting contributions from the team based on individual research Tasks presented in the Technical Narrative. Peer-reviewed journal publications and conference presentations will be prepared at the discretion of the PI and team but discussed collectively at the regular project meetings.

\subsection*{Intellectual Property}
\noindent Project members are employees of Affiliation I, Affiliation II, and others. Intellectual property rights are retained by each employee and employer, per individual employment contract.

\end{document}
