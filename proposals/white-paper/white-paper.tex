%@TheDoctorRAB
%Standard white paper/preproposal format when there is a page limit
%neup.bst designed to use numbered citations in order of appearance, short author list
\documentclass[11pt,letterpaper]{article}
\usepackage[lmargin=1in,rmargin=1in,tmargin=1in,bmargin=1in]{geometry}
\usepackage[pagewise]{lineno} %line numbering
\usepackage{setspace}
\usepackage{ulem} %strikethrough
\usepackage{xcolor,colortbl} %change font color
\usepackage{graphicx}
\usepackage{filecontents}
\usepackage{tablefootnote}
\usepackage{subfig}
\usepackage[yyyymmdd]{datetime}
\renewcommand{\dateseparator}{.}
\graphicspath{{../img/}}
\setcounter{secnumdepth}{5} %set subsection to nth level
\usepackage{times}
\usepackage{enumitem}
\usepackage{multirow}
%\usepackage[labelfont=bf]{caption}

\usepackage[singlelinecheck=false]{caption}
\captionsetup[table]{skip=0pt} %sets a space after table caption
\captionsetup[figure]{skip=0pt,labelformat={default},labelsep=period,name={Fig.}} %sets space above caption, 'figure' format

\usepackage{wrapfig}
\setlength{\intextsep}{0.10mm}
\setlength{\columnsep}{0.10mm}

\usepackage[numbers]{natbib} %use 'numbers' for numbered citations; 'round' for () instead [] for inline citations
\setlength{\bibsep}{0pt} %sets space between references
\renewcommand{\bibsection}{} %suppresses large 'references' heading
\renewcommand\bibpreamble{\vspace{-0.2\baselineskip}} %sets spacing after heading if not using default references heading

\newcommand{\edit}[1]{\textcolor{blue}{#1}} %shortcut for changing font color on revised text
\newcommand{\fn}[1]{\footnote{#1}} %shortcut for footnote tag
\newcommand*\sq{\mathbin{\vcenter{\hbox{\rule{.3ex}{.3ex}}}}} %makes a small square as a separator $\sq$
\newcommand{\x}{\cellcolor{lightgray}} %use to shade in table cell
\newcommand{\sk}[1]{\sout{#1}} %shortcut for strikethrough

\newcolumntype{L}[1]{>{\raggedright\let\newline\\\arraybackslash\hspace{0pt}}p{#1}} %uses \raggedright with p{} in table column

%\paragraph and \subparagraph modifiers
% no indent for each
% after \z@ pads whitespace on the top
% em sets the distance after heading to text horizonally

\makeatletter
\renewcommand\paragraph{%
    \@startsection{paragraph}{4}{\z@ }{0.55\baselineskip}{-1em}
    {\normalfont \normalsize \bfseries}}%

\makeatletter
\renewcommand\subparagraph{%
    \@startsection{subparagraph}{5}{\z@ }{0.45\baselineskip}{-1em}
    {\normalfont \normalsize \itshape }}%

\makeatletter
\renewcommand\subsection{%
    \@startsection{subsection}{2}{\z@ }{0.75\baselineskip}{0.25\baselineskip}
    { \large \bfseries}}%

\usepackage{fancyhdr}
\pagestyle{fancy}
\fancyhf{} %move page number to bottom right
\renewcommand{\headrulewidth}{0.5pt} %turn off line in header
\lhead{\scriptsize Name}
\chead{\scriptsize Title, subject, agency, etc.}
\rhead{\scriptsize \today}
\rfoot{\thepage}

\begin{filecontents}{references.bib}
\end{filecontents}

\begin{document}

{\centering 
    \textbf{Title\\
    Call\\
    }
    Name (PI) - Affiliation\\
    Name (co-PI) - Affiliation 
\par
}
\vspace{\baselineskip}

\noindent\textbf{Summary of the Proposed Project.} 
\\

\noindent\textbf{Motivation.}
\\

\noindent\textbf{Overall Objective.}
\\

\noindent\textbf{Importance and Relevance to Objectives.}
\\

\noindent\textbf{Logical Path, Work Scope, Description of Tasks.} The Logical Path and the proposed Work Scope is described below in terms of the defined Tasks for the project.
\\

\noindent\textbf{Task I. Title.}
\\

\noindent\textbf{Task II. Title.}
\\

\noindent\textbf{Task III. Title.}
\\

\noindent\textbf{Task N. Title.}
\\

\begin{table}[h!]
    \centering
    \caption*{\textbf{Timeframe for Execution of Proposed Project. Schedule, Roles, and Responsibilities.}}
    \begin{tabular}{|l|c|c|c|c|c|c|c|c|c|c|c|c|}
        \hline
        \multicolumn{1}{|c|}{\multirow{2}{*}{\textbf{TASKS}}}& 
        \multicolumn{4}{|c|}{\textbf{Y1}}&
        \multicolumn{4}{|c|}{\textbf{Y2}}&
        \multicolumn{4}{|c|}{\textbf{Y3}}\\
        \cline{2-13}
        &
        \textbf{Q1}& 
        \textbf{Q2}& 
        \textbf{Q3}& 
        \textbf{Q4}& 
        \textbf{Q1}& 
        \textbf{Q2}& 
        \textbf{Q3}& 
        \textbf{Q4}& 
        \textbf{Q1}& 
        \textbf{Q2}& 
        \textbf{Q3}& 
        \textbf{Q4}\\
        \hline
        I. 
        &\x\textbf{I}
        & 
        & 
        & 
        & 
        & 
        & 
        & 
        & 
        & 
        & 
        & \\
        \hline
        II. 
        &\x\textbf{II} 
        & 
        & 
        & 
        & 
        & 
        & 
        & 
        & 
        & 
        & 
        & \\
        \hline
        III. 
        &\x\textbf{III} 
        & 
        & 
        & 
        & 
        & 
        & 
        & 
        & 
        & 
        & 
        & \\
        \hline
        IV. 
        &\x\textbf{IV}
        & 
        & 
        & 
        & 
        & 
        & 
        & 
        & 
        & 
        & 
        & \\
        \hline
        V. 
        &\x\textbf{V} 
        & 
        & 
        & 
        & 
        & 
        & 
        & 
        & 
        & 
        & 
        & \\
        \hline
        VI. 
        &\x\textbf{VI} 
        & 
        & 
        & 
        & 
        & 
        & 
        & 
        & 
        & 
        & 
        & \\
        \hline
        VII. 
        &\x\textbf{VII} 
        & 
        & 
        & 
        & 
        & 
        & 
        & 
        & 
        & 
        & 
        & \\
        \hline
    \end{tabular}
    \label{tab-timeframe}
\end{table}

\noindent\textbf{References}
\bibliographystyle{neup}
\setlength{\bibhang}{0pt}
\bibliography{references}

%wrap figure around text
%move to wherever in the text 
%{l} fixes it on the left margin
%{L} floats
%\begin{wrapfigure}{l}{0.x\textwidth}
%       \includegraphics[width=0.(x-2)\textwidth]{}
%    \caption{}
%    \label{fig-}
%\end{wrapfigure}

\end{document}
