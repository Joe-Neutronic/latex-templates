\documentclass[11pt,letterpaper]{article}
\usepackage[lmargin=1in,rmargin=1in,tmargin=1in,bmargin=1in]{geometry}
\usepackage[pagewise]{lineno} %line numbering
\usepackage{setspace}
\usepackage{ulem} %strikethrough
\usepackage{xcolor,colortbl} %change font color
\usepackage{graphicx}
\usepackage{filecontents}
\usepackage{tablefootnote}
\usepackage{subfig}
\usepackage[yyyymmdd]{datetime}
\renewcommand{\dateseparator}{.}
\graphicspath{{../img/}}
\setcounter{secnumdepth}{5} %set subsection to nth level
\usepackage{times}
\usepackage{enumitem}
\usepackage{float}
\usepackage{multirow}
%\usepackage[labelfont=bf]{caption}

\usepackage[singlelinecheck=false]{caption}
\captionsetup[table]{skip=0pt} %sets a space after table caption
\captionsetup[figure]{skip=0pt,labelformat={default},labelsep=period,name={Fig.}} %sets space above caption, 'figure' format

\usepackage{wrapfig}
\setlength{\intextsep}{0.10mm}
\setlength{\columnsep}{0.10mm}

\usepackage[numbers]{natbib} %use 'numbers' for numbered citations; 'round' for () instead [] for inline citations
\setlength{\bibsep}{0pt} %sets space between references
\renewcommand{\bibsection}{} %suppresses large 'references' heading
\renewcommand\bibpreamble{\vspace{-0.2\baselineskip}} %sets spacing after heading if not using default references heading

\newcommand{\edit}[1]{\textcolor{blue}{#1}} %shortcut for changing font color on revised text
\newcommand{\fn}[1]{\footnote{#1}} %shortcut for footnote tag
\newcommand*\sq{\mathbin{\vcenter{\hbox{\rule{.3ex}{.3ex}}}}} %makes a small square as a separator $\sq$
\newcommand{\x}{\cellcolor{lightgray}} %use to shade in table cell

\usepackage{fancyhdr}
\pagestyle{fancy}
\fancyhf{} %move page number to bottom right
\renewcommand{\headrulewidth}{0pt} %set line thickness in header
%\lhead{\scriptsize Name}
%\chead{\scriptsize Title, subject, agency, etc.}
%\rhead{\scriptsize \today}
\rfoot{\thepage}

\begin{document}

{\centering 
    \textbf{BENEFIT OF COLLABORATORS\\
    Title\\
    Call\\
    }
    Name (PI) - Affiliation\\
    Name (co-PI) - 
\par
}

\vspace{0.5\baselineskip}

\noindent\textbf{Summary of the Proposed Project.} 

\vspace{0.5\baselineskip}

\noindent\textbf{Motivation.} 

\vspace{0.5\baselineskip}

\noindent\textbf{Research Focus.} The N workscopes in this project are - 
\begin{enumerate}[topsep=0pt,itemsep=-1ex,partopsep=1ex,parsep=1ex]
    \item\textit{Title (Lead - )}
        \begin{itemize}[topsep=-1ex,itemsep=-1ex,partopsep=1ex,parsep=1ex]
            \item 
            \item 
        \end{itemize}
    \item\textit{Crosscutting Issues (Full team)}
        \begin{itemize}[topsep=-1ex,itemsep=-1ex,partopsep=1ex,parsep=1ex]
            \item Identify lessons learned
            \item Develop future research pathways
        \end{itemize}
\end{enumerate}

\vspace{0.5\baselineskip}

\vspace{\baselineskip}

\noindent\textbf{Team.} This research will facilitate collaboration among current industry experts with extensive \textit{fill in phrase about team expertise.} The project team consists of the following members. The contribution of each is noted. The corresponding facilities are briefly described thereafter.

\vspace{0.5\baselineskip}

\noindent\textbf{Prof./Dr. Name} (Ph.D., Degree Granting Institution) is a \textit{title} at \textit{Affiliation}, etc. His/Her/Their research focus is in \textit{area of expertise}. Prof./Dr. Name is the PI of this project and will be responsible for progress and achievement of all objectives, tasks, and deliverables. He/She/They has expertise in \textit{topics contained in text of narrative} from \textit{where/how this expertise was acquired}. He/She/They teach \textit{relevant courses} at \textit{Affiliation}.

\vspace{0.5\baselineskip}

\noindent\textbf{Prof./Dr. Name} (Ph.D., Degree Granting Institution) is a \textit{title} at \textit{Affiliation}, etc. His/Her/Their research focus is in \textit{area of expertise}. Prof./Dr. Name is the co-PI of this project and will contibute expertise in \textit{related project tasks}. He/She/They also is/was \textit{related experience} from \textit{where/how this expertise was acquired}. He/She/They teach \textit{relevant courses} at \textit{Affiliation}.

\vspace{0.5\baselineskip}

\noindent\textbf{Important Research Facility/Equipment} is \textit{description as relevant to project.}

\vspace{0.5\baselineskip}

\noindent\textbf{Work Project Documentation, Quality Control.} The co-PIs have practiced a QA approach, consistent with that established by the DOE Technical Integration Office, per NEUP website and any additional requirements deemed necessary during contracting. All projects undergo safety review by experts. We have followed this QA expectation in all ongoing awards. We abide by current and pending information on nuclear-related federal funding contained in Form 3204. The Mandatory, Go/No-go, requirements will be coordinated with UI’s Office of Sponsored Research, the general counsel and Technology Transfer. The PIs assume responsibility as the POC and follow these requirements: 1) Commitment to reporting and budget requirements, 2) 10CFR851 Worker Safety and Health Program, 3) Export Control and 4) Quality Assurance.

\end{document}
