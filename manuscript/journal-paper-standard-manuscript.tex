%@TheDoctorRAB
%Use for original, revised, clean manuscripts
\documentclass[11pt,a4paper]{article}
\usepackage[lmargin=1in,rmargin=1in,tmargin=1in,bmargin=1in]{geometry}
\usepackage[pagewise]{lineno} %line numbering
\usepackage{setspace}
\usepackage{ulem} %strikethrough - do not \sout{\cite{}}
\usepackage{xcolor,colortbl} %change font color
\usepackage{graphicx}
\usepackage{filecontents}
\usepackage{tablefootnote}
\usepackage{footnotehyper}
%\usepackage{subfig}
\usepackage[yyyymmdd]{datetime} %date format
\renewcommand{\dateseparator}{.}
\graphicspath{{../img/}} %path to graphics
\setcounter{secnumdepth}{5} %set subsection to nth level
\usepackage{times}
\usepackage{tabto} %general tabbed spacing
\usepackage{longtable} %need to put label at top under caption then \\ - use spacing
\usepackage[stable,hang,flushmargin]{footmisc} %footnotes in section titles and no indent
\usepackage[round,semicolon]{natbib} %use 'numbers' for numbered citations; 'round' for () instead [] for inline citations
%\usepackage[numbers,sort&compress]{natbib} %use 'numbers' for numbered citations; 'round' for () instead [] for inline citations; nsf.bst
\usepackage{enumitem}
\usepackage{boldline}
\usepackage{makecell}
\usepackage{booktabs}
\usepackage{amssymb}
\usepackage{amsmath}
\usepackage{physics}
\usepackage{tabularx}
\usepackage{multirow}
\usepackage{lscape}
\usepackage{array}
\usepackage{caption}
\usepackage{subcaption}
\usepackage[labelfont=bf]{caption}
\usepackage{chngcntr}
%\usepackage{hyperref}
\usepackage{sectsty}
\usepackage{textcomp}
\usepackage{lastpage}
\usepackage[toc,page]{appendix}
\usepackage[figure,table]{totalcount}
\usepackage[acronym,nomain,nonumberlist]{glossaries}
\makenoidxglossaries

\usepackage[singlelinecheck=false]{caption}
\captionsetup[table]{skip=7pt} %sets a space after table caption
\captionsetup[figure]{skip=7pt,labelformat={default},labelsep=period} %sets space above caption, 'figure' format

\usepackage{wrapfig}
\setlength{\intextsep}{0.20mm}
\setlength{\columnsep}{0.20mm}

%\usepackage{xr} %for revisions - will cross reference from one file to here
%\externaldocument{/path/to/auxfilename} %aux file needed

\newcommand{\edit}[1]{\textcolor{blue}{#1}} %shortcut for changing font color on revised text
\newcommand{\fn}[1]{\footnote{#1}} %shortcut for footnote tag
\newcommand*\sq{\mathbin{\vcenter{\hbox{\rule{.3ex}{.3ex}}}}} %makes a small square as a separator $\sq$
\newcommand{\sk}[1]{\sout{#1}} %shortcut for strikethrough
\newcommand{\x}{\cellcolor{lightgray}} %use to shade in table cell
\newcommand{\acf}{\acrfull} %full acronym
\newcommand{\acl}{\acrlong} %long acronym
\newcommand{\acs}{\acrshort} %short acronym

\newcolumntype{L}[1]{>{\raggedright\let\newline\\\arraybackslash\hspace{0pt}}p{#1}} %uses \raggedright with p{} in table column

\makeatletter
\renewcommand\tableofcontents{%
    \@starttoc{toc}%
}
\makeatother

\makeatletter
\renewcommand\listoffigures{%
    \@starttoc{lof}%
}
\makeatother

\makeatletter
\renewcommand\listoftables{%
    \@starttoc{lot}%
}
\makeatother

%\setlength{\bibsep}{0pt} %sets space between references
%\renewcommand{\bibsection}{} %suppresses large 'references' heading
%\renewcommand\bibpreamble{\vspace{-0.2\baselineskip}} %sets spacing after heading if not using default references heading

\usepackage{fancyhdr}
\pagestyle{fancy}
\fancyhf{} %move page number to bottom right
%\renewcommand{\headrulewidth}{0pt} %set line thickness in header; uncomment as is to remove line
\lhead{\scriptsize Manuscript ID - AHP NHRES}
\chead{\scriptsize \textit{Redfoot et al. - Draft Manuscript}}
\rhead{\scriptsize \today}
\rfoot{\thepage}

\begin{filecontents}{references.bib} 
    @misc{
    ,
    author = {{}},
    title = {{}},
    year = {},
    }
    @article{
    ,
    author = {{}},
    journal = {},
    pages = {},
    title = {{}},
    volume = {},
    year = {}
    }
    @conference{
    ,
    author = {},
    title = {{}},
    year = { },
    organization = { },
    address = {}
    }    
    @book{
    ,
    author = {{}},
    title = {{}},
    publisher = {},
    year = {},
    isbn = {}
    }    
    @incollection{
    ,
    author = {{}},
    title = {{}},
    booktitle = {},
    publisher = {},
    year = {},
    note = {}
    }  
\end{filecontents}

%\newacronym{}{}{}

\begin{document}

\begin{titlepage}
    \title{Title}
    \author{
        \textsuperscript{a}Author 1\textsuperscript{*}, 
        \textsuperscript{b}Author 2, 
        \textsuperscript{c}Author 3
        \\ \\ \\
        \textsuperscript{a}Affiliation 1\\ 
        \\ \\
        \textsuperscript{b}Affiliation 2\\ 
        \\ \\
        \textsuperscript{c}Affiliation 3\\
        \\ \\ \\
        \textsuperscript{*}corresponding author email
    }
\clearpage %not have page number on title page
\maketitle
\vspace*{\fill}
\begin{flushright}{
        \noindent Number of pages - \pageref{LastPage} \\
        \noindent Number of tables - \totaltables \\
        \noindent Number of figures - \totalfigures
}
\end{flushright}
\thispagestyle{empty} %start with page number 1 on second page
\end{titlepage}

\onehalfspacing %linespacing
\linenumbers %toggle line numbers
\pagewiselinenumbers %reset line numbers on new page
\modulolinenumbers[3] %line numbering interval

\section*{Abstract} \label{sec-abstract}

\newpage

\printnoidxglossary

\newpage

\section{Introduction} \label{sec-introduction}
\subsection{Motivation} \label{sec-motivation}
\subsection{Goals} \label{sec-goals}

\newpage

\section{Background} \label{sec-background}

\newpage

\section{Methodology} \label{sec-methodology}

\newpage

\section{Results} \label{sec-results}

\newpage

\section{Discussion} \label{sec-discussion}

\newpage

\section{Future work} \label{sec-future-work}

\newpage

\section{Summary remarks} \label{sec-summary-remarks}

\newpage 

\section*{Acknowledgements}

\newpage

\bibliographystyle{standard}
\setlength{\bibhang}{0pt}
\bibliography{references}

\newpage

\section*{Tables}

{%
\let\oldnumberline\numberline%
\renewcommand{\numberline}{\tablename~\oldnumberline}%
\listoftables%
}

\newpage

\begin{table}[h!]
    \centering
    \caption{Title}
        \begin{tabular}{|c|c|c|c|}
            \hline
            A&
            B&
            C&
            D\\
            \hline
            X&X&X&X\\
            \hline
        \end{tabular}
    \label{tab-density}
\end{table}

\newpage 

\begin{spacing}{1}
\begin{longtable}{|c|c|c|c|c|}
    \caption{Title}
    \label{tab-label-name} \\
    \hline
    \multicolumn{1}{|c|}{A}&
    \multicolumn{1}{c|}{B}&
    \multicolumn{1}{c|}{C}&
    \multicolumn{1}{c|}{D}&
    \multicolumn{1}{c|}{E}\\
    \hline
    \endfirsthead
    \multicolumn{5}{c}{{\tablename\ \thetable{} - continued}} \\
    \hline
    \multicolumn{1}{|c|}{A}&
    \multicolumn{1}{c|}{B}&
    \multicolumn{1}{c|}{C}&
    \multicolumn{1}{c|}{D}&
    \multicolumn{1}{c|}{E}\\
    \hline
    \endhead
    \hline
    \endfoot
    \hline
    \endlastfoot
    X&X&X&X&X\\
    \hline
\end{longtable}
\end{spacing}

\newpage

\section*{Figures}

{%
\let\oldnumberline\numberline%
\renewcommand{\numberline}{\figurename~\oldnumberline}%
\listoffigures%
}

\newpage

%\begin{figure}[h!]
%    \centering
%    \includegraphics[width=0.X\textwidth]{.png}
%    \caption{}
%    \label{fig-label-name}
%\end{figure}

\end{document}
